\documentclass[11pt]{article}

% ========== CONFIGURACIÓN BÁSICA ==========
\usepackage[letterpaper, margin=2cm]{geometry}
\usepackage[spanish,es-tabla]{babel}
\usepackage[utf8]{inputenc}
\usepackage[T1]{fontenc}

% ========== PACKAGES MATEMÁTICOS ==========
\usepackage{amsmath, amssymb, amsfonts}
\usepackage{mathtools}

% ========== PACKAGES GRÁFICOS ==========
\usepackage{graphicx}
\usepackage{xcolor}
\usepackage{tikz}
\usepackage{float}
\usepackage{placeins}

% ========== PACKAGES FORMATO ==========
\usepackage{fancyhdr}
\usepackage{titlesec}
\usepackage{enumitem}
\usepackage{array}
\usepackage{tabularx}
\usepackage{booktabs}
\usepackage{parskip}
\usepackage{multicol}
\usepackage{tcolorbox}
\tcbuselibrary{breakable,skins}

% ========== PACKAGES ESPECÍFICOS SEÑALES ==========
\usepackage{circuitikz}
\usepackage{pgfplots}
\pgfplotsset{compat=1.18}

% ========== COLORES PUC ==========
\definecolor{pucblue}{RGB}{0, 56, 101}
\definecolor{pucgold}{RGB}{198, 146, 20}
\definecolor{pucgray}{RGB}{88, 89, 91}
\definecolor{lightblue}{RGB}{230, 242, 255}
\definecolor{lightgreen}{RGB}{232, 245, 233}
\definecolor{practicecolor}{RGB}{76, 175, 80}

% ========== CONFIGURACIÓN DE PÁGINA ==========
\setlength{\headheight}{50pt}
\setlength{\headsep}{15pt}
\setlength{\footskip}{25pt}

% ========== HEADER Y FOOTER ==========
\pagestyle{fancy}
\fancyhf{}

% Header simplificado para guías
\fancyhead[L]{%
  \begin{minipage}{0.15\textwidth}
    \includegraphics[width=1.5cm]{logo-uc.pdf}
  \end{minipage}%
  \begin{minipage}{0.85\textwidth}
    \small\color{pucblue}
    \textbf{Pontificia Universidad Católica de Chile}\\
    \textbf{IEE2103 - Señales y Sistemas}
  \end{minipage}
}

% Footer
\fancyfoot[L]{\color{pucgray}\small Guía de Ejercicios}
\fancyfoot[C]{\color{pucgray}\small Página \thepage}
\fancyfoot[R]{\color{pucgray}\small Prof. Patricio de la Cuadra}

% ========== COMANDOS PERSONALIZADOS ==========
% Funciones especiales de señales (mismas que en prueba)
\newlength{\widthfontline}
\setlength{\widthfontline}{0.12ex}
\newcommand{\step}{\mbox{\rule{0.3ex}{0cm}%
                        \rule{\widthfontline}{1.0ex}%
                        \hspace{-\widthfontline}%
                        \rule[1.0ex]{1.5ex}{\widthfontline}}}
\newcommand{\shah}{\mbox{\rule{0.3ex}{0cm}%
                        \rule{\widthfontline}{1.5ex}%
                        \rule{0.7ex}{\widthfontline}%
                        \rule{\widthfontline}{1.5ex}%
                        \rule{0.7ex}{\widthfontline}%
                        \rule{\widthfontline}{1.5ex}%
                        \hspace{\widthfontline}}}
\newcommand{\triang}{\wedge}
\newcommand{\sqcap}{\sqcap}

% ========== COMANDOS PARA CONFIGURACIÓN ==========
\newcommand{\configurarguia}[6]{
  \newcommand{\tituloGuia}{#1}
  \newcommand{\unidadGuia}{#2}
  \newcommand{\fechaGuia}{#3}
  \newcommand{\profesorGuia}{#4}
  \newcommand{\semestreGuia}{#5}
  \newcommand{\descripcionGuia}{#6}
}

% ========== COMANDO PARA TÍTULO DE GUÍA ==========
\newcommand{\tituloGuiaPrincipal}{%
  \begin{center}
    % Título principal
    \begin{tcolorbox}[
      enhanced,
      colback=practicecolor,
      coltext=white,
      boxrule=0pt,
      rounded corners=12pt,
      center title,
      fonttitle=\LARGE\bfseries,
      title=\tituloGuia,
      drop shadow
    ]
    \end{tcolorbox}
    
    \vspace{0.4cm}
    
    % Información de la guía
    \begin{tcolorbox}[
      colback=lightgreen,
      colframe=practicecolor,
      boxrule=1.5pt,
      rounded corners=8pt,
      center
    ]
      \begin{tabularx}{0.9\textwidth}{@{}X@{\hspace{1cm}}X@{}}
        \textbf{Unidad:} \unidadGuia & \textbf{Fecha:} \fechaGuia \\[0.2cm]
        \textbf{Profesor:} \profesorGuia & \textbf{Semestre:} \semestreGuia \\
      \end{tabularx}
      
      \vspace{0.3cm}
      
      \textit{\descripcionGuia}
    \end{tcolorbox}
    
    \vspace{0.3cm}
    
    % Línea divisoria
    \textcolor{practicecolor}{\rule{\textwidth}{1.5pt}}
  \end{center}
  
  \vspace{0.5cm}
}

% ========== COMANDOS PARA DIFICULTAD ==========
\newcommand{\basico}{%
  \begin{tikzpicture}[baseline=(current bounding box.center)]
    \node[draw=green!70!black, fill=green!20, rounded corners=3pt, 
          inner sep=2pt, minimum width=1.5cm] {\small\textbf{Básico}};
  \end{tikzpicture}%
}

\newcommand{\intermedio}{%
  \begin{tikzpicture}[baseline=(current bounding box.center)]
    \node[draw=orange!70!black, fill=orange!20, rounded corners=3pt, 
          inner sep=2pt, minimum width=1.5cm] {\small\textbf{Intermedio}};
  \end{tikzpicture}%
}

\newcommand{\avanzado}{%
  \begin{tikzpicture}[baseline=(current bounding box.center)]
    \node[draw=red!70!black, fill=red!20, rounded corners=3pt, 
          inner sep=2pt, minimum width=1.5cm] {\small\textbf{Avanzado}};
  \end{tikzpicture}%
}

\newcommand{\desafio}{%
  \begin{tikzpicture}[baseline=(current bounding box.center)]
    \node[draw=purple!70!black, fill=purple!20, rounded corners=3pt, 
          inner sep=2pt, minimum width=1.5cm] {\small\textbf{Desafío}};
  \end{tikzpicture}%
}

% ========== ENTORNO PARA EJERCICIOS ==========
\newcounter{ejercicio}
\newenvironment{ejercicio}[3][]{%
  \stepcounter{ejercicio}
  \vspace{0.6cm}
  \noindent
  \begin{tcolorbox}[
    enhanced,
    breakable,
    colback=white,
    colframe=practicecolor,
    boxrule=1pt,
    rounded corners=5pt,
    left=10pt,
    right=10pt,
    top=8pt,
    bottom=8pt
  ]
  \textbf{\color{practicecolor}Ejercicio \theejercicio} 
  \ifx&#1&\else\textbf{ - #1}\fi
  \hfill 
  \textbf{Tiempo:} #2 min \quad #3
  
  \vspace{0.3cm}
}{%
  \end{tcolorbox}
}

% ========== ENTORNOS PARA SECCIONES TEMÁTICAS ==========
\newenvironment{unidadintro}{%
  \begin{tcolorbox}[
    enhanced,
    colback=blue!5,
    colframe=pucblue,
    boxrule=1.5pt,
    rounded corners=8pt,
    title={\textbf{📚 Introducción}},
    fonttitle=\bfseries\color{pucblue}
  ]
}{%
  \end{tcolorbox}
  \vspace{0.5cm}
}

\newenvironment{unidadconceptos}{%
  \begin{tcolorbox}[
    enhanced,
    colback=green!5,
    colframe=green!70!black,
    boxrule=1.5pt,
    rounded corners=8pt,
    title={\textbf{🔑 Conceptos Clave}},
    fonttitle=\bfseries\color{green!70!black}
  ]
}{%
  \end{tcolorbox}
  \vspace{0.5cm}
}

\newenvironment{unidadformulas}{%
  \begin{tcolorbox}[
    enhanced,
    colback=orange!5,
    colframe=orange!70!black,
    boxrule=1.5pt,
    rounded corners=8pt,
    title={\textbf{📐 Fórmulas Importantes}},
    fonttitle=\bfseries\color{orange!70!black}
  ]
}{%
  \end{tcolorbox}
  \vspace{0.5cm}
}

% ========== COMANDO PARA HINTS Y TIPS ==========
\newcommand{\tip}[1]{%
  \begin{tcolorbox}[
    colback=yellow!10,
    colframe=orange,
    boxrule=1pt,
    rounded corners=3pt,
    left=8pt,
    right=8pt,
    top=5pt,
    bottom=5pt
  ]
  \textbf{💡 TIP:} #1
  \end{tcolorbox}
}

\newcommand{\atencion}[1]{%
  \begin{tcolorbox}[
    colback=red!10,
    colframe=red,
    boxrule=1pt,
    rounded corners=3pt,
    left=8pt,
    right=8pt,
    top=5pt,
    bottom=5pt
  ]
  \textbf{⚠️ ATENCIÓN:} #1
  \end{tcolorbox}
}

% ========== ENTORNO PARA SOLUCIONES ==========
\newif\ifsoluciones
% \solucionestrue  % Descomentar para mostrar soluciones

\newenvironment{solucion}{%
  \ifsoluciones
    \vspace{0.3cm}
    \begin{tcolorbox}[
      enhanced,
      breakable,
      colback=green!5,
      colframe=green!70!black,
      boxrule=1pt,
      rounded corners=5pt,
      left=8pt,
      right=8pt,
      top=8pt,
      bottom=8pt,
      title={\textbf{\color{green!70!black}✅ SOLUCIÓN}},
      fonttitle=\bfseries
    ]
  \fi
}{%
  \ifsoluciones
    \end{tcolorbox}
  \fi
}

% ========== COMANDOS PARA ESPACIOS DE TRABAJO ==========
\newcommand{\espaciotrabajo}[1][5cm]{\vspace{#1}}

% ========== ENTORNO PARA GRUPOS DE EJERCICIOS ==========
\newenvironment{grupoejercicio}[2]{%
  \vspace{0.8cm}
  \begin{tcolorbox}[
    enhanced,
    colback=pucblue!5,
    colframe=pucblue,
    boxrule=2pt,
    rounded corners=8pt,
    title={\textbf{#1}},
    fonttitle=\Large\bfseries\color{pucblue}
  ]
  \textit{#2}
  \end{tcolorbox}
  \vspace{0.3cm}
}{%
  \vspace{0.5cm}
}

% ========== COMANDO PARA OBJETIVOS DE APRENDIZAJE ==========
\newcommand{\objetivos}[1]{%
  \begin{tcolorbox}[
    enhanced,
    colback=pucgold!10,
    colframe=pucgold,
    boxrule=1.5pt,
    rounded corners=8pt,
    title={\textbf{🎯 OBJETIVOS DE APRENDIZAJE}},
    fonttitle=\bfseries\color{pucgold}
  ]
  \begin{itemize}[leftmargin=20pt, itemsep=5pt]
    #1
  \end{itemize}
  \end{tcolorbox}
  \vspace{0.5cm}
}

% ========== INICIO DEL DOCUMENTO ==========
\begin{document}

% ========== CONFIGURAR ESTA GUÍA ==========
\configurarguia{
  Guía de Ejercicios - Convolución y Sistemas LTI  % Título
}{
  Unidad 2: Sistemas Continuos  % Unidad
}{
  Septiembre 2024  % Fecha
}{
  Patricio de la Cuadra  % Profesor
}{
  2024-2  % Semestre
}{
  Ejercicios de práctica para dominar los conceptos de convolución, respuesta al impulso y análisis de sistemas LTI continuos.  % Descripción
}

% ========== TÍTULO PRINCIPAL ==========
\tituloGuiaPrincipal

% ========== OBJETIVOS DE APRENDIZAJE ==========
\objetivos{
  \item Aplicar la operación de convolución tanto gráfica como analíticamente
  \item Determinar la respuesta al impulso de sistemas LTI
  \item Analizar propiedades de sistemas: linealidad, invariancia temporal, causalidad
  \item Resolver problemas de sistemas descritos por ecuaciones diferenciales
  \item Interpretar físicamente el comportamiento de sistemas LTI
}

% ========== CONCEPTOS CLAVE ==========
\begin{unidadconceptos}
  \textbf{Sistemas LTI:} Sistemas que cumplen con las propiedades de \textit{linealidad} e \textit{invariancia temporal}.
  
  \textbf{Convolución:} Operación matemática que describe la salida de un sistema LTI: $y(t) = x(t) * h(t) = \int_{-\infty}^{\infty} x(\tau)h(t-\tau)d\tau$
  
  \textbf{Respuesta al impulso:} Salida del sistema cuando la entrada es un impulso $\delta(t)$.
\end{unidadconceptos}

% ========== FÓRMULAS IMPORTANTES ==========
\begin{unidadformulas}
  \begin{align}
    \text{Convolución: } y(t) &= \int_{-\infty}^{\infty} x(\tau)h(t-\tau)d\tau \\
    \text{Propiedades: } x(t) * h(t) &= h(t) * x(t) \text{ (conmutativa)} \\
    x(t) * [\delta(t-t_0)] &= x(t-t_0) \text{ (propiedad del impulso)} \\
    \frac{d}{dt}[x(t) * h(t)] &= \frac{dx(t)}{dt} * h(t) = x(t) * \frac{dh(t)}{dt}
  \end{align}
\end{unidadformulas}

% ========== GRUPO 1: EJERCICIOS BÁSICOS ==========
\begin{grupoejercicio}{Propiedades de Sistemas LTI}{Ejercicios fundamentales para verificar linealidad e invariancia temporal}

\begin{ejercicio}[Verificación de propiedades]{15}{\basico}
  Determine si los siguientes sistemas son lineales y/o invariantes en el tiempo:
  
  \begin{enumerate}[label=\alph*)]
    \item $y(t) = 2x(t) + 3$
    \item $y(t) = x^2(t)$
    \item $y(t) = \cos(t) \cdot x(t)$
    \item $y(t) = x(2t)$
  \end{enumerate}
  
  \tip{Para verificar linealidad, compruebe que $T[ax_1(t) + bx_2(t)] = aT[x_1(t)] + bT[x_2(t)]$}
  
  \espaciotrabajo[6cm]
  
  \begin{solucion}
    \begin{enumerate}[label=\alph*)]
      \item \textbf{No lineal} (término constante), \textbf{Invariante}
      \item \textbf{No lineal} (cuadrático), \textbf{Invariante}
      \item \textbf{Lineal}, \textbf{No invariante} (coeficiente depende del tiempo)
      \item \textbf{Lineal}, \textbf{No invariante} (escalamiento temporal)
    \end{enumerate}
  \end{solucion}
\end{ejercicio}

\begin{ejercicio}[Respuesta al escalón]{20}{\intermedio}
  Un sistema LTI tiene respuesta al impulso $h(t) = e^{-2t}u(t)$. Encuentre:
  
  \begin{enumerate}[label=\alph*)]
    \item La respuesta al escalón unitario $s(t) = u(t) * h(t)$
    \item La respuesta a la entrada $x(t) = e^{-t}u(t)$
  \end{enumerate}
  
  \espaciotrabajo[8cm]
  
  \begin{solucion}
    \begin{enumerate}[label=\alph*)]
      \item $s(t) = u(t) * e^{-2t}u(t) = \int_0^t e^{-2\tau}d\tau = \frac{1-e^{-2t}}{2}u(t)$
      \item $y(t) = e^{-t}u(t) * e^{-2t}u(t) = (e^{-t} - e^{-2t})u(t)$
    \end{enumerate}
  \end{solucion}
\end{ejercicio}

\end{grupoejercicio}

% ========== GRUPO 2: CONVOLUCIÓN GRÁFICA ==========
\begin{grupoejercicio}{Convolución Gráfica}{Desarrollo de intuición visual para la operación de convolución}

\begin{ejercicio}[Convolución de pulsos rectangulares]{25}{\intermedio}
  Dadas las señales:
  \begin{align}
    x(t) &= \begin{cases} 1 & 0 \leq t \leq 2 \\ 0 & \text{otro caso} \end{cases} \\
    h(t) &= \begin{cases} 1 & 0 \leq t \leq 1 \\ 0 & \text{otro caso} \end{cases}
  \end{align}
  
  \begin{enumerate}[label=\alph*)]
    \item Grafique $x(t)$ y $h(t)$
    \item Grafique $h(-\tau)$ y $h(t-\tau)$ para $t = 0.5, 1.5, 2.5$
    \item Calcule y grafique $y(t) = x(t) * h(t)$
  \end{enumerate}
  
  \espaciotrabajo[10cm]
  
  \begin{solucion}
    La convolución resulta en:
    $y(t) = \begin{cases}
      0 & t < 0 \\
      t & 0 \leq t \leq 1 \\
      1 & 1 < t \leq 2 \\
      3-t & 2 < t \leq 3 \\
      0 & t > 3
    \end{cases}$
  \end{solucion}
\end{ejercicio}

\begin{ejercicio}[Convolución con exponencial]{30}{\avanzado}
  Calcule analíticamente la convolución $y(t) = x(t) * h(t)$ donde:
  \begin{align}
    x(t) &= e^{-t}u(t) \\
    h(t) &= \delta(t) + 2\delta(t-1)
  \end{align}
  
  \tip{Use la propiedad del impulso: $x(t) * \delta(t-t_0) = x(t-t_0)$}
  
  \espaciotrabajo[6cm]
  
  \begin{solucion}
    $y(t) = e^{-t}u(t) + 2e^{-(t-1)}u(t-1) = e^{-t}u(t) + 2e \cdot e^{-t}u(t-1)$
  \end{solucion}
\end{ejercicio}

\end{grupoejercicio}

% ========== GRUPO 3: SISTEMAS Y ECUACIONES DIFERENCIALES ==========
\begin{grupoejercicio}{Sistemas y Ecuaciones Diferenciales}{Análisis de sistemas descritos por ecuaciones diferenciales}

\begin{ejercicio}[Sistema RC]{35}{\avanzado}
  Un circuito RC tiene la ecuación diferencial:
  $RC\frac{dy(t)}{dt} + y(t) = x(t)$
  
  \begin{enumerate}[label=\alph*)]
    \item Encuentre la respuesta al impulso $h(t)$
    \item Si $x(t) = u(t)$, encuentre $y(t)$ usando convolución
    \item Verifique el resultado resolviendo directamente la ecuación diferencial
  \end{enumerate}
  
  \atencion{Recuerde las condiciones iniciales en reposo para sistemas causales}
  
  \espaciotrabajo[10cm]
  
  \begin{solucion}
    \begin{enumerate}[label=\alph*)]
      \item $h(t) = \frac{1}{RC}e^{-t/(RC)}u(t)$
      \item $y(t) = (1 - e^{-t/(RC)})u(t)$
      \item El resultado coincide usando condiciones iniciales en reposo
    \end{enumerate}
  \end{solucion}
\end{ejercicio}

\begin{ejercicio}[Sistema masa-resorte-amortiguador]{40}{\desafio}
  Un sistema mecánico se describe por:
  $\frac{d^2y(t)}{dt^2} + 3\frac{dy(t)}{dt} + 2y(t) = x(t)$
  
  \begin{enumerate}[label=\alph*)]
    \item Encuentre las raíces características
    \item Determine la respuesta al impulso
    \item ¿El sistema es estable? Justifique
  \end{enumerate}
  
  \espaciotrabajo[8cm]
  
  \begin{solucion}
    \begin{enumerate}[label=\alph*)]
      \item Raíces: $s_1 = -1$, $s_2 = -2$
      \item $h(t) = (2e^{-t} - e^{-2t})u(t)$
      \item Estable (todas las raíces tienen parte real negativa)
    \end{enumerate}
  \end{solucion}
\end{ejercicio}

\end{grupoejercicio}

% ========== PROBLEMAS DESAFÍO ==========
\begin{grupoejercicio}{Problemas Desafío}{Para estudiantes que buscan un reto adicional}

\begin{ejercicio}[Convolución con funciones especiales]{45}{\desafio}
  Calcule la convolución:
  $\int_{-\infty}^{\infty} \text{sinc}(t-\tau) \cdot \text{rect}(\tau) d\tau$
  
  donde $\text{sinc}(t) = \frac{\sin(\pi t)}{\pi t}$ y $\text{rect}(t) = \begin{cases} 1 & |t| \leq 1/2 \\ 0 & |t| > 1/2 \end{cases}$
  
  \tip{Considere usar propiedades de la transformada de Fourier si las conoce}
  
  \espaciotrabajo[8cm]
\end{ejercicio}

\end{grupoejercicio}

% ========== RECURSOS ADICIONALES ==========
\begin{tcolorbox}[
  enhanced,
  colback=pucblue!5,
  colframe=pucblue,
  boxrule=1.5pt,
  rounded corners=8pt,
  title={\textbf{📖 RECURSOS ADICIONALES}},
  fonttitle=\bfseries\color{pucblue}
]
\begin{itemize}
  \item \textbf{Libro texto:} "Señales y Sistemas: Fundamentos Matemáticos", Pablo Alvarado Moya, Cap. 3-4
  \item \textbf{Simulaciones:} Use MATLAB/Python para visualizar convoluciones
  \item \textbf{Videos:} Revise las grabaciones de clases en Canvas
  \item \textbf{Ayudantías:} Asista a las sesiones de resolución de ejercicios
\end{itemize}
\end{tcolorbox}

% ========== PIE DE PÁGINA ==========
\vfill
\begin{center}
  \begin{tikzpicture}
    \draw[practicecolor, line width=1.5pt] (0,0) -- (\textwidth,0);
  \end{tikzpicture}
  \vspace{0.3cm}
  
  \textcolor{pucgray}{\large ¡Practica hace al maestro!}\\
  \textcolor{pucgray}{\small Consulta en ayudantías o Discord si tienes dudas}
\end{center}

\end{document}