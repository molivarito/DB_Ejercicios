\documentclass[11pt]{article}

% ========== CONFIGURACIÓN BÁSICA ==========
\usepackage[letterpaper, margin=2cm]{geometry}
\usepackage[spanish,es-tabla]{babel}
\usepackage[utf8]{inputenc}
\usepackage[T1]{fontenc}

% ========== PACKAGES MATEMÁTICOS ==========
\usepackage{amsmath, amssymb, amsfonts}
\usepackage{mathtools}

% ========== PACKAGES GRÁFICOS ==========
\usepackage{graphicx}
\usepackage{xcolor}
\usepackage{tikz}
\usepackage{float}
\usepackage{placeins}

% ========== PACKAGES FORMATO ==========
\usepackage{fancyhdr}
\usepackage{titlesec}
\usepackage{enumitem}
\usepackage{array}
\usepackage{tabularx}
\usepackage{booktabs}
\usepackage{parskip}
\usepackage[none]{hyphenat}

% ========== PACKAGES ESPECÍFICOS SEÑALES ==========
\usepackage{circuitikz}  % Para diagramas de circuitos
\usepackage{pgfplots}    % Para gráficos de señales
\pgfplotsset{compat=1.18}

% ========== COLORES PUC ==========
\definecolor{pucblue}{RGB}{0, 56, 101}
\definecolor{pucgold}{RGB}{198, 146, 20}
\definecolor{pucgray}{RGB}{88, 89, 91}
\definecolor{lightblue}{RGB}{230, 242, 255}
\definecolor{answerred}{RGB}{220, 50, 47}

% ========== CONFIGURACIÓN DE PÁGINA ==========
\setlength{\headheight}{60pt}
\setlength{\headsep}{20pt}
\setlength{\footskip}{30pt}

% ========== HEADER Y FOOTER ==========
\pagestyle{fancy}
\fancyhf{}

% Header
\fancyhead[L]{%
  \begin{minipage}{0.15\textwidth}
    \includegraphics[width=1.8cm]{logo-uc.pdf}
  \end{minipage}%
  \begin{minipage}{0.85\textwidth}
    \small\color{pucblue}
    \textbf{Pontificia Universidad Católica de Chile}\\
    Escuela de Ingeniería\\
    Departamento de Ingeniería Eléctrica\\
    \textbf{IEE2103 - Señales y Sistemas}
  \end{minipage}
}

% Footer
\fancyfoot[C]{\color{pucgray}\small Página \thepage}
\fancyfoot[R]{\color{pucgray}\small Prof. Patricio de la Cuadra}

% ========== COMANDOS PERSONALIZADOS ==========
% Funciones especiales de señales
\newlength{\widthfontline}
\setlength{\widthfontline}{0.12ex}
\newcommand{\step}{\mbox{\rule{0.3ex}{0cm}%
                        \rule{\widthfontline}{1.0ex}%
                        \hspace{-\widthfontline}%
                        \rule[1.0ex]{1.5ex}{\widthfontline}}}
\newcommand{\shah}{\mbox{\rule{0.3ex}{0cm}%
                        \rule{\widthfontline}{1.5ex}%
                        \rule{0.7ex}{\widthfontline}%
                        \rule{\widthfontline}{1.5ex}%
                        \rule{0.7ex}{\widthfontline}%
                        \rule{\widthfontline}{1.5ex}%
                        \hspace{\widthfontline}}}
\newcommand{\triang}{\wedge}
\newcommand{\sqcap}{\sqcap}

% Comando para espacios de respuesta
\newcommand{\respuesta}[1][4cm]{\vspace{#1}}

% Comando para recuadros de puntos
\newcommand{\puntos}[1]{%
  \begin{tikzpicture}[baseline=(current bounding box.center)]
    \node[draw=pucblue, fill=lightblue, rounded corners=3pt, 
          inner sep=3pt, minimum width=1.5cm] {\textbf{#1 pts}};
  \end{tikzpicture}%
}

% ========== CONFIGURACIÓN DE SECCIONES ==========
\titleformat{\section}[hang]
{\Large\bfseries\color{pucblue}}
{\thesection.}{0.5em}{}
[\vspace{0.2cm}\titlerule[0.8pt]]

\titleformat{\subsection}[hang]
{\large\bfseries\color{pucblue}}
{\thesubsection.}{0.5em}{}

% ========== ENTORNOS PERSONALIZADOS ==========
% Entorno para instrucciones
\newenvironment{instrucciones}{%
  \begin{tcolorbox}[
    colback=lightblue,
    colframe=pucblue,
    boxrule=1pt,
    rounded corners=5pt,
    left=10pt,
    right=10pt,
    top=8pt,
    bottom=8pt
  ]
  \textbf{\color{pucblue}INSTRUCCIONES:}
  \begin{enumerate}[leftmargin=20pt]
}{%
  \end{enumerate}
  \end{tcolorbox}
}

% Para usar tcolorbox
\usepackage{tcolorbox}
\tcbuselibrary{breakable}

% ========== INFORMACIÓN DE LA PRUEBA ==========
\newcommand{\configurarprueba}[7]{
  \newcommand{\nombrePrueba}{#1}
  \newcommand{\fechaPrueba}{#2}
  \newcommand{\tiempoPrueba}{#3}
  \newcommand{\profesorPrueba}{#4}
  \newcommand{\semestrePrueba}{#5}
  \newcommand{\totalPuntos}{#6}
  \newcommand{\instruccionesPrueba}{#7}
}

% ========== COMANDO PRINCIPAL ==========
\newcommand{\tituloPrueba}{%
  \begin{center}
    % Título principal
    \begin{tcolorbox}[
      colback=pucblue,
      coltext=white,
      boxrule=0pt,
      rounded corners=8pt,
      center title,
      fonttitle=\Large\bfseries,
      title=\nombrePrueba
    ]
    \end{tcolorbox}
    
    \vspace{0.5cm}
    
    % Información de la prueba
    \begin{tabularx}{\textwidth}{@{}X@{\hspace{2cm}}X@{}}
      \textbf{Fecha:} \fechaPrueba & \textbf{Tiempo:} \tiempoPrueba \\
      \textbf{Profesor:} \profesorPrueba & \textbf{Semestre:} \semestrePrueba \\
      \multicolumn{2}{c}{\textbf{Puntaje Total:} \totalPuntos puntos}
    \end{tabularx}
    
    \vspace{0.3cm}
    
    % Línea divisoria
    \textcolor{pucblue}{\rule{\textwidth}{1.5pt}}
  \end{center}
  
  \vspace{0.5cm}
}

% ========== ENTORNO PARA EJERCICIOS ==========
\newcounter{ejercicio}
\newenvironment{ejercicio}[2][]{%
  \stepcounter{ejercicio}
  \vspace{0.8cm}
  \noindent
  \begin{tcolorbox}[
    enhanced,
    breakable,
    colback=white,
    colframe=pucblue,
    boxrule=1.5pt,
    rounded corners=5pt,
    attach boxed title to top left={xshift=10pt, yshift=-5pt},
    boxed title style={
      colback=pucblue,
      coltext=white,
      rounded corners=3pt,
      boxrule=0pt
    },
    title={\textbf{Ejercicio \theejercicio} \ifx&#1&\else - #1\fi \hfill \puntos{#2}}
  ]
}{%
  \end{tcolorbox}
}

% ========== ESPACIOS PARA DATOS ESTUDIANTE ==========
\newcommand{\datosEstudiante}{%
  \begin{center}
    \begin{tabularx}{0.8\textwidth}{|X|X|}
      \hline
      \textbf{Nombre:} \rule{6cm}{0.4pt} & \textbf{RUT:} \rule{4cm}{0.4pt} \\[0.5cm]
      \hline
      \textbf{Sección:} \rule{2cm}{0.4pt} & \textbf{Nota:} \rule{2cm}{0.4pt} \\[0.5cm]
      \hline
    \end{tabularx}
  \end{center}
  \vspace{0.5cm}
}

% ========== COMANDO PARA SOLUCIONES ==========
\newif\ifsoluciones
% \solucionestrue  % Descomentar para mostrar soluciones

\newenvironment{solucion}{%
  \ifsoluciones
    \begin{tcolorbox}[
      colback=red!5,
      colframe=answerred,
      boxrule=1pt,
      rounded corners=3pt,
      left=5pt,
      right=5pt,
      top=5pt,
      bottom=5pt,
      breakable
    ]
    \textbf{\color{answerred}SOLUCIÓN:}\\
  \fi
}{%
  \ifsoluciones
    \end{tcolorbox}
  \fi
}

% ========== INICIO DEL DOCUMENTO ==========
\begin{document}

% ========== CONFIGURAR ESTA PRUEBA ==========
\configurarprueba{
  Interrogación 1  % Nombre
}{
  15 de Septiembre, 2024  % Fecha
}{
  90 minutos  % Tiempo
}{
  Patricio de la Cuadra  % Profesor
}{
  2024-2  % Semestre
}{
  42  % Puntaje total
}{
  % Instrucciones (se definen abajo)
}

% ========== TÍTULO Y DATOS ==========
\tituloPrueba

\datosEstudiante

% ========== INSTRUCCIONES ==========
\begin{instrucciones}
  \item Está permitido usar como máximo \textbf{dos carillas} de apuntes manuscritos.
  \item Responda cada pregunta en \textbf{hojas separadas}, utilizando solamente lápiz de tinta.
  \item \textbf{Coloque su nombre y RUT} en todas las hojas que ocupe. ¡Hágalo ahora!
  \item Las hojas sin nombre \textbf{NO serán corregidas}.
  \item No se permiten consultas de \textbf{NINGÚN tipo}.
  \item Si considera que falta información, haga una suposición razonable y \textbf{exprésela claramente}.
  \item Muestre \textbf{claramente todos los pasos} de desarrollo.
\end{instrucciones}

% ========== EJERCICIOS ==========

\begin{ejercicio}[Manipulación de señales complejas]{6}
  Dada la señal $x(t)=\sqrt{2}(1+j)e^{j\pi/4}e^{(-1+j2\pi)t}$:
  
  \begin{enumerate}[label=\alph*)]
    \item Bosqueje Re$\{x(t)\}$
    \respuesta[2.5cm]
    
    \item Bosqueje Im$\{x(t)\}$
    \respuesta[2.5cm]
    
    \item Bosqueje $x(t+2)+x^*(t+2)$
    \respuesta[3cm]
  \end{enumerate}
  
  \textbf{Parte 2:} Muestre que:
  \[(1-e^{j\alpha})=2\sin\left(\frac{\alpha}{2}\right)e^{j[(\alpha-\pi)/2]}\]
  
  \respuesta[3cm]
  
  \begin{solucion}
    Para la parte (a), primero simplificamos la expresión...
    [Solución completa aquí]
  \end{solucion}
\end{ejercicio}

\begin{ejercicio}[Propiedades de sistemas LTI]{6}
  Determine si los siguientes sistemas son \textbf{lineales} y/o \textbf{invariantes en el tiempo}:
  
  \begin{enumerate}[label=\alph*)]
    \item $y(t)=\int_{0}^{t}x(\tau) d\tau$
    \respuesta[2.5cm]
    
    \item $y(t)=x(t)+t$
    \respuesta[2.5cm]
    
    \item $y(t)=1/x(t)$
    \respuesta[3cm]
  \end{enumerate}
  
  \begin{solucion}
    Para cada sistema, verificamos linealidad probando...
    [Solución completa aquí]
  \end{solucion}
\end{ejercicio}

\begin{ejercicio}[Integral con funciones especiales]{6}
  Resuelva la siguiente integral:
  \[\int_{-\infty}^{\infty} \shah(kx) \triang(x) dx, \quad k>0\]
  
  \respuesta[8cm]
  
  \begin{solucion}
    Utilizando la propiedad del cedazo...
    [Solución completa aquí]
  \end{solucion}
\end{ejercicio}

\newpage

\begin{ejercicio}[Sistema LTI causal]{6}
  Un sistema LTI causal responde con $y(t)$ a una entrada $x(t)$:
  \[x(t)=u(t+\frac{1}{2})-u(t-\frac{1}{2})\]
  \[y(t)=x(2t+\frac{1}{2})(2t+1)+x(2t-\frac{1}{2})(1-2t)\]
  
  \begin{enumerate}[label=\alph*)]
    \item Grafique las funciones $x(t)$ e $y(t)$
    \respuesta[4cm]
    
    \item ¿Cuál es la respuesta del sistema a la entrada $x_2(t)=x(\frac{t-1}{2})$?
    \respuesta[3cm]
    
    \item ¿Cuál es la respuesta $y_3(t)$ al escalón unitario $u(t)$?
    \respuesta[3cm]
  \end{enumerate}
\end{ejercicio}

\begin{ejercicio}[Convolución gráfica]{6}
  Dadas las funciones $h(t)=3e^{-3t}u(t)$ y $f(t)=2(u(t)-u(t-1))$:
  
  Encuentre la convolución entre $h(t)$ y $f(t)$ \textbf{gráfica y analíticamente}.
  
  \respuesta[10cm]
\end{ejercicio}

\begin{ejercicio}[Identificación de convoluciones]{6}
  \textbf{[Incluir las figuras proporcionadas]}
  
  Indique cuáles de las funciones dadas deben ser convolucionadas para obtener las funciones mostradas en las figuras de salida.
  
  \respuesta[6cm]
\end{ejercicio}

% ========== PIE DE PÁGINA FINAL ==========
\vfill
\begin{center}
  \textcolor{pucgray}{\rule{0.5\textwidth}{0.5pt}}\\
  \small\textcolor{pucgray}{¡Éxito en su evaluación!}
\end{center}

\end{document}