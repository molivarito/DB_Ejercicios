\documentclass[letterpaper, 12pt]{article}
\usepackage[spanish, es-tabla]{babel} %%Paquete español para mac
\usepackage[utf8]{inputenc} %% Para unicode
\usepackage{graphicx} %% Para incluir figuras
\usepackage{ifpdf}
%\usepackage{mathabx}
\usepackage{xcolor}
\DeclareGraphicsExtensions{.pdf}
\usepackage{fullpage}
\usepackage{amssymb}
\usepackage[cmex10]{amsmath}
%\usepackage{amsmath}
\usepackage[all]{xy}
\usepackage{MnSymbol}
\setcounter{totalnumber}{5}
\renewcommand{\textfraction}{0.1}
\usepackage{float}
\decimalpoint
\usepackage{url}
\graphicspath{Enunciados/}
\usepackage{hyperref}
\usepackage{biblatex} 
\usepackage{csquotes}
\addbibresource{Citas.bib}
%\newcommand{\sha}{\textls{\text{l}\!\_\!\text{l}\!\_\!\text{l}}}
\usepackage[OT2,T1]{fontenc}
\DeclareSymbolFont{cyrletters}{OT2}{wncyr}{m}{n}
\DeclareMathSymbol{\sha}{\mathalpha}{cyrletters}{"58}
\hypersetup{colorlinks=false,bookmarksopen=true,linkbordercolor={1 1 1}}


\newif\ifanswers
%%%%%%%%%%%%%%%%%%%%%%   OCULTAR SOLUCIONES  %%%%%%%%%%%%%%%%%%%%%% 

% comentar la siguiente línea permite ocultar las soluciones de los ejercicios

\answerstrue

%%%%%%%%%%%%%%%%%%%%%%   ESTRUCTURA DE  EJERCICIOS  %%%%%%%%%%%%%%%%%%%%%% 

% Los ejercicios están separados por secciones, siguiendo la siguiente estructura:

%  \subsection*{Tema a tratar}

%       \begin{enumerate}
%               % Cada ejercicio se asocia a un ítem
%               \item << Enunciado del ejercicio >>
%               
%               \ifanswers % condición "if" para ocultar las soluciones
%               {\color{red} % Permite que el texto de las soluciones se vea en rojo 
%
%                 << SOLUCIÓN DEL EJERCICIO >>
%
%               }
%               \fi %
%               \item <<Otro ejercicio>>
%       \end{enumerate}
%




\usepackage{enumerate} 
\setlength{\parindent}{0pt}
\begin{document}
%%%%%%%%%%%%%%%%%%%%%%%%%%
%%%%%%   ENCABEZADO  %%%%%
%%%%%%%%%%%%%%%%%%%%%%%%%%
\vspace*{-1cm}
\includegraphics[width=2cm]{logo_uc_medio.jpg}
\vspace*{-2cm}

\hspace*{2cm}
 \begin{tabular}{l}
  {\ Pontificia Universidad Católica de Chile}\\
  {\ Escuela de Ingeniería}\\
  {\ Departamento de Ingeniería Eléctrica}\\
  {\ IEE2103 – Señales y Sistemas}\
 \end{tabular}
 \hfill 
\vspace*{1mm}
\begin{center}
{\LARGE\bf Guía: Capítulo 1 (Versión 3)}\\
\vspace*{2mm}
Matthew Helm - mhelm@uc.cl\\
Martín Peña  - martinpena@uc.cl\\
Ana M. Bolados - ana.bolados@uc.cl\\
\textit{Se utilizan ejercicios de MIT OpenCourseWare y ayudantías de Carlos Castillo}


\end{center}
\hrule\vspace*{2pt}\hrule

\setlength{\parindent}{0pt}
% ***********************************************
% ***********************************************
\vspace*{10pt}
\section*{Ejercicios por tema}


\subsection*{Números complejos}

\begin{enumerate}

%%%%%%%%%%%%%%%%%%%%%%  NÚMEROS COMPLEOS - P1  %%%%%%%%%%%%%%%%%%%%%% 


    \item Se tiene el número complejo $2z = 1 +i\sqrt{3}$ y $2w = \sqrt{2}-i\sqrt{2}$ ahora calcule usando la forma polar:
    \begin{enumerate}
        \item $z\cdot w$
        \item $z/w$
        \item $zz^{*},\frac{1}{2} (z + z^{*}), \frac{1}{2i} (z - z^{*})$
        \item Magnitud y fase de $(z\cdot w)^{*}$
    \end{enumerate}
    \ifanswers
    {\color{red} \textbf{Solución:} \textit{Resuelta en ayudantía}}
    \fi
    
%%%%%%%%%%%%%%%%%%%%%%  NÚMEROS COMPLEOS - P2  %%%%%%%%%%%%%%%%%%%%%% 

    \item Demuestre que para todo $z \in \mathbb{C}$  se cumple que:
    \begin{itemize}
        \item $\Re{\{z\}} = \frac{z + z^*}{2}$
        \item $j\Im{\{z\}}  = \frac{z-z^*}{2}$
    \end{itemize}

    \ifanswers
    {\color{red}\textbf{Solución:} 

    
    Tomamos un complejo cualquiera $z = \sigma + j \omega$\\
    $$z + z^* = \sigma + j\omega + \sigma -j\omega = 2\sigma $$
    $$\frac{z + z^*}{2} = \sigma$$
    Similarmente:
    $$z-z^* = \sigma +j\omega - \sigma + j\omega = 2j\omega$$
    $$\frac{z - z^*}{2} = j \omega$$}
    \fi
    %%%%%%%%%%%%%%%%%%%%%%  NÚMEROS COMPLEOS - P3  %%%%%%%%%%%%%%%%%%%%%% 

    \item Demuestre que 
    \begin{align*}
        1-e^{j\alpha} = 2\sin \left(\frac{\alpha}{2}\right)e^{j(\alpha-\pi)/2}
    \end{align*}
    
    \ifanswers
    {\color{red} \textbf{Solución:}

    Aplicando la identidad de Euler a la parte derecha de la igualdad se obtiene:
    \begin{align*}
        1-e^{j\alpha} &= 2\sin\left(\frac{\alpha}{2}\right)(\cos\left(\frac{\alpha}{2}-\frac{\pi}{2}\right)+j\sin\left(\frac{\alpha}{2}-\frac{\pi}{2}\right))
    \end{align*}
    Recordando que $\cos(\theta-\pi/2) = \sin(\theta)$ y $\sin(\theta-\pi/2)=-\cos(\theta)$ se tendrá que
    \begin{align*}
        1-e^{j\alpha} &= 2\sin\left(\frac{\alpha}{2}\right)(\sin\left(\frac{\alpha}{2}\right)-j\cos\left(\frac{\alpha}{2}\right)) \\
        1-e^{j\alpha} &= 2\sin\left(\frac{\alpha}{2}\right)^2-2j\sin\left(\frac{\alpha}{2}\right)\cos\left(\frac{\alpha}{2}\right)
    \end{align*}
    Lo anterior se puede desarrollar con las siguientes identidades trigonométricas: $\sin(\theta)^2=\frac{1}{2}(1-\cos(2\theta))$ y $\sin(2\theta) = 2\sin(\theta)\cos(\theta)$. Así,
    \begin{align*}
        1-e^{j\alpha} &= 2\left(\frac{1-\cos(\alpha)}{2}\right)-j\sin(\alpha) \\
        &= \ 1 - (\cos(\alpha)+j\sin(\alpha)) = 1 - e^{j\alpha},
    \end{align*}
    lo que demuestra lo pedido.}
    \fi
%%%%%%%%%%%%%%%%%%%%%%  NÚMEROS COMPLEOS - P4  %%%%%%%%%%%%%%%%%%%%%% 

    \item Considere la señal $x(t) = \sqrt{2}(1+j)e^{j\pi/4}e^{(-1+j 2\pi)t} $. Determine:
    \begin{enumerate}
        \item $\Re\{x(t)\}$
        \item $\Im\{x(t)\}$ 
        \item $x(t+2)+x^{*}(t+2)$
    \end{enumerate}

    \ifanswers
    {\color{red}\textbf{Solución:}
    
    Primero, puede ser conveniente desarrollar $x(t)$ en su forma polar:
    \begin{align*}
        1+j &= \sqrt{2}e^{j\pi/4} \\
        \sqrt{2}(1+j)e^{j\pi/4} &= 2e^{j\pi/2}= 2 j \\
        e^{(-1+2j\pi)t} &= e^{-t} e^{2j\pi t} \\
        x(t) &= 2j e^{-t}(\cos(2\pi t)+j\sin(2\pi t)) \\
        x(t) &= -2e^{-t}\sin(2\pi t) + 2je^{-t} \cos(2\pi t)
    \end{align*}
    \begin{enumerate}
        \item Del resultado anterior, es evidente que $\Re\{x(t)\}=-2e^{-t}\sin(2\pi t)$.
        \item De manera similar, $\Im\{x(t)\}=2e^{-t}\cos(2\pi t)$.
        \item Un número complejo más su conjugado corresponde a dos veces la parte real (evaluado en $t^{*}=t+2$ para este caso). Así, $x(t+2)+x^{*}(t+2) = -4e^{(-t-2)}\sin(2\pi(t+2))=-4e^{-(t+2)}\sin(2\pi t)$.
    \end{enumerate}}
    \fi

    %%%%%%%%%%%%%%%%%%%%%%  NÚMEROS COMPLEOS - P5  %%%%%%%%%%%%%%%%%%%%%% 

    \item Determinar el conjunto de puntos que verifican las condiciones siguientes: 
    \begin{enumerate}
        \item \(|z-2+3i| = 1\)
        \item \(\Re (z^* - i) = 2\)
    \end{enumerate}

    \ifanswers
    {\color{red}\textbf{Solución:}
    \begin{enumerate}
        \item Sea \(z = x + iy\), entonces \(z-2 + 3i = (x - 2) + i(y + 3) \) y \(|z-2+3i| = \sqrt{(x-2)^2 + (y+3)^2} = 1\), se cumple que los puntos que satisfacen esta igualdad son los puntos pertenecientes a la circunferencia de radio 1 y centro en \((2, -3)\).

        \item Sustituyendo \( z = x + iy \), se obtiene \( \Re(z^* - i) = \Re(x - iy - i) = x \). La ecuación \( x = 2 \) corresponde a la recta vertical que pasa por el punto \( (2,0) \).
    \end{enumerate}
    
    }
    \fi

    %%%%%%%%%%%%%%%%%%%%%%  NÚMEROS COMPLEOS - P6  %%%%%%%%%%%%%%%%%%%%%% 

    \item Hallar todos los valores de \( (2i)^{1/2} \) y calcular la raíz principal.

    \ifanswers
  {\color{red}  \textbf{Solución:}
  
    Sea \( w = 2i \). Entonces:
    \[
    w = 2i = 2e^{i\pi/2} = 2e^{i(\pi/2 + 2k\pi)}, \quad k = 0, 1, 2, \dots
    \]
    Por tanto, \( z = w^{1/2} = \sqrt{2}e^{i(\pi/4 + k\pi)} \). Damos valores a \( k \) hasta que encontramos el primer repetido:
    \[
    z_{k=0} = \sqrt{2}e^{i\pi/4}
    \]
    \[
    z_{k=1} = \sqrt{2}e^{i(\pi/4 + \pi)} = \sqrt{2}e^{5i\pi/4}
    \]
    \[
    z_{k=2} = \sqrt{2}e^{i(\pi/4 + 2\pi)} = z_{k=0}
    \]
    Vemos que para \( k = 2 \) se obtiene la misma raíz que para \( k = 0 \) ya que \( e^{i2\pi} \) implica dar una vuelta completa. Así pues, únicamente existen dos soluciones distintas, \( \sqrt{2}e^{i\pi/4} \) y \( \sqrt{2}e^{5i\pi/4} \). La raíz principal es la que corresponde a \( k = 0 \), es decir, \( \sqrt{2}e^{i\pi/4} \).
   }
    \fi

    %%%%%%%%%%%%%%%%%%%%%%  NÚMEROS COMPLEOS - P7  %%%%%%%%%%%%%%%%%%%%%% 

    \item Resolver la ecuación \(\cos z = 2\)

    \ifanswers
  {\color{red}  \textbf{Solución:}
  
    Obviamente, si $z$ fuera real, la ecuación no tendría solución, puesto que el coseno toma valores entre -1 y 1, pero esto no tiene por qué ser cierto en los $\mathbb{C}$.
    
    Empleamos la definición de \( \cos z \):
    \[
    e^{iz} + e^{-iz} = 4
    \]
    Introducimos ahora la variable \( w \equiv e^{iz} \):
    \[
    w + \frac{1}{w} = 4 \implies w^2 - 4w + 1 = 0
    \]
    Las soluciones de esta ecuación cuadrática son \( w = (2 \pm \sqrt{3})e^{2in\pi} \) con \( n = 0, \pm1, \pm2, \dots \). Escribimos \( z = x + iy \) y hallamos \( x \) e \( y \):
    \[
    w = e^{iz} \implies (2 \pm \sqrt{3})e^{2in\pi} = e^{ix}e^{-y}
    \]
    Comparamos ambos miembros de la ecuación:
    \[
    2 \pm \sqrt{3} = e^{-y} \implies y = -\ln(2 \pm \sqrt{3})
    \]
    \[
    e^{2in\pi} = e^{ix} \implies x = 2n\pi
    \]
    Por tanto, las soluciones de la ecuación son:
    \[
    z = 2n\pi - i\ln(2 + \sqrt{3})
    \]
    y
    \[
    z = 2n\pi - i\ln(2 - \sqrt{3})
    \]
    donde \( n = 0, \pm1, \pm2, \dots \)
   }
    \fi

    %%%%%%%%%%%%%%%%%%%%%%  NÚMEROS COMPLEOS - P8  %%%%%%%%%%%%%%%%%%%%%% 

    \item Ordene de menor a mayor las partes reales de los siguientes números complejos: \( (-1)^i \), \( \sinh(\pi i) \), \( i\ln\sqrt{-1+i} \). Tome el valor principal del logaritmo cuando proceda.

    \ifanswers
  {\color{red}  \textbf{Solución:}
  
    Para calcular la parte real, expresamos los números complejos en forma cartesiana:
    \[
    (-1)^i = e^{i\ln(-1)} = e^{i(\ln |1| + i\pi)} = e^{-\pi}
    \]
    \[
    \sinh(\pi i) = i\sin\pi = 0
    \]
    \[
    i\ln\sqrt{-1+i} = \frac{i}{2}\ln(-1+i) = \frac{i}{2}\left(\ln\sqrt{2} + i\frac{3\pi}{4}\right) = -\frac{3\pi}{8} + i\frac{1}{4}\ln 2
    \]
    Luego, \( \Re[i\ln\sqrt{-1+i}] < \Re [\sinh(\pi i)] < \Re [(-1)^i] \).
   }
    \fi
    
    
    
\end{enumerate}

\subsection*{Señales y Sistemas}

\begin{enumerate}

%%%%%%%%%%%%%%%%%%%%%%  SEÑALES Y SISTEMAS - P1  %%%%%%%%%%%%%%%%%%%%%% 

    \item Considere las señales de entrada
    \begin{align*}
        x(t) &= \cos\left(\frac{2\pi t}{3}\right) + 2 \sin\left(\frac{16 \pi t}{3}\right) \\
        y(t) &= \sin(\pi t)
    \end{align*}
        El sistema se modela por la relación $z(t)=x(t)y(t)$, donde $z(t)$ es la señal salida.
        Exprese $z(t)$ como una combinación lineal de exponenciales complejas de la forma
        \begin{align*}
            z(t) &= \sum_{k} c_k e^{j k (2\pi/T)},
        \end{align*}
        identificando el valor de $T$ y los coeficientes $c_k$.
  

    \ifanswers
  {\color{red}  \textbf{Solución:}

    Las señales $x(t)$ e $y(t)$ se pueden descomponer como una suma de exponenciales complejas, lo que lleva a
    \begin{align*}
        x(t) &= \frac{1}{2} e^{j(2\pi t/3)} + \frac{1}{2} e^{-2\pi t/3}+\frac{e^{16\pi t/3}}{j}-\frac{e^{-16\pi t /3}}{j} \\
        y(t) &= \frac{e^{j\pi t}}{2j}-\frac{e^{-j\pi t}}{2j}
    \end{align*}
    Luego, $z(t)$ será
    \begin{align*}
        z(t) &= \frac{1}{4j}e^{j5\pi t /3} - \frac{1}{4j}e^{-j\pi t /3}+\frac{1}{4j}e^{j\pi t/3}-\frac{1}{4j}e^{-j5\pi t/3} - \frac{1}{2}e^{j19\pi t/3} + \frac{1}{2}e^{j13\pi t/3}-\frac{1}{2}e^{-j13\pi t/3}-\frac{1}{2}e^{-j19\pi t/3}
    \end{align*}

    Se puede notar que todas las exponenciales complejas son potencias de $e^{j\pi t/3}$, por lo que $T = \frac{2\pi}{(\pi/3)} = 6 \ [s]$. Los coeficientes son los que multiplican cada exponencial compleja.}
    \fi
%%%%%%%%%%%%%%%%%%%%%%  SEÑALES Y SISTEMAS - P2  %%%%%%%%%%%%%%%%%%%%%% 

    \item Sea la señal $x(t) = \cos{(\omega_x(t+\tau_x)+\theta_x)}$:

    \begin{enumerate}
        \item Determine la frecuencia en Hz y el período de $x(t)$ para los casos:
        \begin{enumerate}[i)]
            \item $\omega_x = \pi/3$, $\tau_x = 0$, $\theta_x = 2\pi$
            \item $\omega_x = 3\pi/4$, $\tau_x = 1/2$, $\theta_x = \pi/4$
            \item $\omega_x = 3/4$, $\tau_x = 1/2$, $\theta_x = 1/4$
        \end{enumerate}
    \item Ahora considere la señal $y(t) = \sin{(\omega_y(t+\tau_y)+\theta_y)}$ y para los siguientes casos evalúe en qué casos $x(t) = y(t)$:
    \begin{enumerate}[i)]
        \item $\omega_x = \pi/3$, $\tau_x = 0$, $\theta_x = 2\pi$ y $\omega_y = \pi/3$, $\tau_y = 1$, $\theta_y = -\pi/3$
    
        \item $\omega_x = 3\pi/4$, $\tau_x = 1/2$, $\theta_x = \pi/4$ y $\omega_y = 11\pi/4$, $\tau_y = 1$, $\theta_y = 3\pi/8$
    
        \item $\omega_x = 3/4$, $\tau_x = 1/2$, $\theta_x = 1/4$ y $\omega_y = 3/4$, $\tau_y = 1$, $\theta_y = 3/8$
    \end{enumerate}
    \end{enumerate}
        \ifanswers
       {\color{red} \textbf{Solución:}
       
    .
       \begin{enumerate}[a)]
           \item    Primero notamos que $\omega = 2\pi f$ con $f$ la frecuencia en Hz y similarmente $T = 1/f$. Notamos que la frecuencia de la señal no depende de $\tau$ in $\theta$.
           \begin{enumerate}[i)]
               \item $f = \frac{\pi/3}{2\pi} = 1/6 $ y $ T = 6$
               \item $f = 3/8$ y $T = 8/3$
               \item $f = 3/(8\pi)$ y $T = 8 \pi/3 $
           \end{enumerate}
           
           %\fi
        \item Para que las señales sean iguales debe complirse que $\omega_x = \omega_y$ y además $\omega_x \tau_x +\theta_x = \omega_y \tau_y + \theta_y + 2\pi k$ con $k$ cualquier entero.
        \begin{enumerate}[i)]
            \item $x(t) = y(t)$
            \item $x(t) \neq y(t)$
            \item $x(t) \neq y(t)$
        \end{enumerate}
       \end{enumerate}
    
       
       }
    \fi

    %%%%%%%%%%%%%%%%%%%%%%  SEÑALES Y SISTEMAS - P3  %%%%%%%%%%%%%%%%%%%%%% 

       \item Tenemos el sistema:
    $$ g(x) = \int_{-\infty}^{x}f(t) \sum_{n = -\infty}^{\infty}sinc(t - n)dt$$
    
    Calcule $\frac{d g(x)}{dx}$ si $f(x) = \text{l}\!\_\!\text{l}\!\_\!\text{l}(x) $
    
    \textit{Hint: Piense qué valores toma $sinc(t - n)$ cuando $n$ es un entero y en particular en $n = 0$.}



    
    \ifanswers
    {\color{red}
    \textbf{Solución:}


    \textbf{Cada sinc(x) desfasado vale 1 solo en su origen y para todos los demás enteros vale cero.}\\ Además recordando la propiedad del muestreo $\delta_a \cdot f(x) = f(a)$  podemos escribir $$g(x) = \int_{-\infty}^{x}\text{l}\!\_\!\text{l}\!\_\!\text{l}(x) = \int_{-\infty}^{x} \sum_{n = -\infty}^{\infty}\delta(x-n)$$. Podemos recordar que la integral del impulso se puede escribir como un escalón por lo que:$$g(x) =  \sum_{n = -\infty}^{\infty}u(x-n)$$. De esta forma podemos volver a derviar el sistema para llegar a que $$g(x)^{'} =\text{l}\!\_\!\text{l}\!\_\!\text{l}(x) $$}

    \fi
%%%%%%%%%%%%%%%%%%%%%%  SEÑALES Y SISTEMAS - P4  %%%%%%%%%%%%%%%%%%%%%% 

    \item Exprese las siguientes ecuaciones como sumas infinitas de polinomios y determine para qué valores de $|x|$ la serie converge:
    \begin{enumerate}
        \item  $p_1(x) = \frac{1}{1-x}$
        \item $p_2(x) = \frac{1}{(1-x)^2}$
    \end{enumerate}
    \textit{HINT: Puede ser útil usar series de Taylor.}

         \ifanswers
    {\color{red}
    \textbf{Solución:}
    \begin{enumerate}
        \item Esta pregunta se puede resolver sencillamente sin usar series de Taylor, ya que sabemos que toda serie geométrica tiene solución dada por:
        $$\sum_{n = 0}^{+\infty}a^n=\frac{1}{1-a}$$
        Por lo tanto:
        $$p_1(x) = \frac{1}{1-x} = 1+x+x^2+x^3+x^4+...$$
        Además converge siempre con $|x|<1$.

        \item Se puede resolver de manera similar notando que esa es la solución para la serie $\sum_{n = 0}^{+\infty}(n+1)a^n$. Sin embargo es más razonable hacerla mediante series de Taylor centrada en cero:
        $$f(x) = f(0)+f^{\prime}(0)(x)+\frac{f^{\prime \prime}(0)(x^2)}{2!}+...+\frac{f^{(n)}(0)(x^n)}{n!}$$

        Al desarrollar y reemplazar llegamos a:
        $$p_2(x) = 1 + 2x + 3x^2+ 4x^3+5x^4+...$$

        Similarmente esto converge con $\lim_{n \to \infty } \{(n+1)x^n\} = 0$  es decir $|x| <1$.
    \end{enumerate}
    }
    \fi
%%%%%%%%%%%%%%%%%%%%%%  SEÑALES Y SISTEMAS - P5  %%%%%%%%%%%%%%%%%%%%%% 

    \item Determine si los siguientes sistemas son causales:
    \begin{enumerate}
        \item $g(t) = f(t-2)$
        \item $g(t) = f(-t)$ 
        \item $g(t) = \int_{-5}^{5} f(\tau) d\tau$ 
        \item $g(t) = \frac{f^2(t)}{df(t)/dt}$
    \end{enumerate}

    \ifanswers
    {\color{red}    \textbf{Solución:}
    \begin{enumerate}
        \item La salida en el instante $t$ requiere conocer la información de la entrada en $t-2 < t$ (información pasada), por lo que el sistema es \textbf{causal}.
        \item Si $t<0$, se requiere conocer la entrada del sistema en $t>0$ (información futura), por lo que el sistema \textbf{no es causal}.
        \item Si se quiere conocer la salida del sistema para $t \in (-\infty,5)$, se requerirá conocer la entrada para valores futuros (hasta $t=5$), por lo que el sistema \textbf{no es causal}.
        \item La salida del sistema para el instante $t$ depende de la derivada de la entrada en el $t$. Como la derivada es un operador local de la función que depende de un límite sobre la entrada en $t^{+}>t$, el sistema es \textbf{no causal}.
    \end{enumerate}}
    
    \fi
%%%%%%%%%%%%%%%%%%%%%%  SEÑALES Y SISTEMAS - P6  %%%%%%%%%%%%%%%%%%%%%% 

    \item Considere un sistema con entrada $x(t)$ y salida $y(t)$, el cual viene dado por:
    \begin{align*}
        y(t) &= \sum_{n=-\infty}^{\infty} x(t) \delta(t-n T)
    \end{align*}
    
    \begin{enumerate}
        \item Identifique (justificadamente) si el sistema es lineal y si el sistema es invariante.
        \item Si el sistema presenta como entrada $x(t) = \cos(2\pi t)$, indique la salida $y(t)$ que se generaría para $T=1, \frac{1}{2}, \frac{1}{4}, \frac{1}{8}$.
    \end{enumerate}
    \ifanswers
    {\color{red}
    \textbf{Solución:}

    \begin{enumerate}
        \item El sistema es lineal porque
        \begin{align*}
            T[ax_1(t)+bx_2(t)] &= \sum_{n=-\infty}^{\infty} [ax_1(t)+bx_2(t)]\delta(t-nT) \\
            &= a \sum_{n=-\infty}^{\infty} x_1(t) \delta(t-n T) + b \sum_{n=-\infty}^{\infty} x_2(t) \delta(t-n T) \\
            &= a T[x_1(t)] + b T[x_2(t)]
        \end{align*}
        Por otro lado, el sistema es variante porque si $x_1(t)=\sin(2\pi t/T)$, la salida será $y_1(t)=0$. Sin embargo, si $x_2(t)=\sin(\frac{2\pi}{T} t+\frac{\pi}{2}) = \cos(\frac{2\pi}{T}t) \neq y_1(t+\frac{\pi}{2})=0$.
        \item La salida está dada por
        \begin{align*}
            y(t) &= \sum_{n=-\infty}^{\infty} \cos(2\pi t) \delta(t-n T)
        \end{align*}

        \begin{figure}[H]
            \centering
            \includegraphics[width=0.6\textwidth]{P13/y_T_1.png}
            \caption{$y(t)$ para $T=1$}
            \label{y1}
        \end{figure}
        \begin{figure}[H]
            \centering
            \includegraphics[width=0.6\textwidth]{P13/y_T_05.png}
            \caption{$y(t)$ para $T=1/2$}
            \label{y1}
        \end{figure}
        \begin{figure}[H]
            \centering
            \includegraphics[width=0.6\textwidth]{P13/y_T_025.png}
            \caption{$y(t)$ para $T=1/4$}
            \label{y1}
        \end{figure}
        \begin{figure}[H]
            \centering
            \includegraphics[width=0.6\textwidth]{P13/y_T_0125.png}
            \caption{$y(t)$ para $T=1/8$}
            \label{y1}
        \end{figure}
    \end{enumerate}
    
    }

        \fi
%%%%%%%%%%%%%%%%%%%%%%  SEÑALES Y SISTEMAS - P7  %%%%%%%%%%%%%%%%%%%%%% 

    \item  Considere el sistema

    $$
    y(t)=x(t-\tau)
    $$
    
    donde $y(t)$ es la salida, $x(t)$ la entrada y $\tau \in \mathbb{R}$.
    
    Determine si el sistema es invertible. En caso positivo, explicite cual sería el sistema inverso. En caso negativo, demuestre que el sistema no es invertible.

    \ifanswers
    {\color{red}
    \textbf{Solución: }

    (a) Observando que el sistema todo lo que hace es desplazar la señal de entrada en una cantidad $\tau$ hacia la derecha, el sistema inverso tiene que ser aquel que desplaza la señal la misma cantidad hacia la izquierda y vuelve a centrar la entrada en su posición original, es decir
        
        $$
        y^{-1}(t)=x(t+\tau) 
        $$
        
        Por lo tanto el sistema es invertible $\rightarrow 1$ pt. Esta es la respuesta corta e intuitiva.
        
        La respuesta formal, es la siguiente. Observamos que el sistema se puede describir por la convolución $y=x * \delta_{\tau}$, donde $\delta_{\tau}$ denota un impulso desfasado en $\tau$. Deducimos entonces que la respuesta al impulso del sistema es
        
        $$
        h=\delta_{\tau}
        $$
        
        Para que el sistema sea invertible, tiene que cumplirse $h * h^{-1}=\delta$. Usando las propiedaes de la convolución del impulso, entonces queda claro que:
        
        $$
        h^{-1}=\delta_{-\tau} 
        $$
        
        dado que
        
        $$
        \delta_{\tau} * \delta_{-\tau}=\delta
        $$
        
        Entonces concluímos que
        
        $$
        y^{-1}(t)=x(t+\tau)
        $$
    
    }
    \fi
%%%%%%%%%%%%%%%%%%%%%%  SEÑALES Y SISTEMAS - P8  %%%%%%%%%%%%%%%%%%%%%% 

    \item Considere el sistema

    $$
    y(t)=x(t) \cos (2 \pi u t)
    $$
    
    donde $y(t)$ es la salida, $x(t)$ la entrada y $u \in \mathbb{R}$.
    
    i. Determine si el sistema es estable.
    
    ii. Determine si el sistema tiene memoria.
    
    iii. Determine si el sistema es lineal.
    
    iv. Determine si el sistema es invariante.
    
    v. Determine si el sistema es causal.

    \ifanswers
    {\color{red}
    \textbf{Solución: }

    i. Si la entrada $x(t)$ es acotada, como $|\cos (2 \pi u t)| \leq 1$ entonces la salida estará acotada, como máximo, al mismo valor de la entrada. Por lo tanto el sistema es estable .
    
    ii. Dado que $y(t)$ solo depende del valor presente de $t$, el sistema no tiene memoria .
    
    iii. Homogeneidad:
    
    $$
    \alpha x(t) \rightarrow \alpha x(t) \cos (2 \pi u t)=\alpha y(t)
    $$
    
    Superposición:
    
    $$
    x_{1}(t)+x_{2}(t) \rightarrow\left(x_{1}(t)+x_{2}(t)\right) \cos (2 \pi u t)=x_{1}(t) \cos (2 \pi u t)+x_{2}(t) \cos (2 \pi u t)=y_{1}(t)+y_{2}(t)
    $$
    
    Por lo tanto, el sistema es lineal .
    
    iv. Invariancia:
    
    $$
    x(t-\tau) \rightarrow x(t-\tau) \cos (2 \pi u t)
    $$
    
    Pero la salida desfasada es
    
    $$
    y(t-\tau)=x(t-\tau) \cos (2 \pi u(t-\tau))
    $$
    
    Como ambas expresiones no son iguales, el sistema es variante.
    
    v. Como el sistema no depende de valores futuros de la entrada el sistema es causal .
    
    Por ejemplo, para el tiempo $t=1$, entonces
    
    $y(1)=x(1) \cos (2 \pi u 1)$.
    
    }
    \fi
\end{enumerate}


\subsection*{Gráficos}

\begin{enumerate}

%%%%%%%%%%%%%%%%%%%%%%  GRÁFICOS - P1  %%%%%%%%%%%%%%%%%%%%%% 

    \item Grafique la función $f^{'}(x)$ si:
    $$ f(x) = [u(x + \frac{\pi}{2}) + u(x)]*\cos^2(x)$$
     \textit{OJO: El operador $*$ es la convolución.}



         
    \ifanswers
        {\color{red}
        \textbf{Solución:}
    
    Comenzamos con la propiedad de derivación de convolución tal que:
         $$f^{\prime}(x) = [u(x + \frac{\pi}{2}) + u(x)]^{\prime}*\cos^2(x) $$ Dado que la derivada del escalón es un $\delta$. Tenemos que:
         $$f^{\prime}(x) = [\delta(x + \frac{\pi}{2}) + \delta(x)]*\cos^2(x) $$ Por propiedad del cedazo esto es lo mismo que: 
         $$f^{\prime}(x) = \cos^2(x + \frac{\pi}{2}) + \cos^2(x) $$      Recordamos que $\cos^2(x + \frac{\pi}{2}) = \sin^2(x)$ Por lo que $$f^{\prime}(x) = \sin^2(x) + \cos^2(x) = 1 $$ El gráfico es constante en 1.  }
    
         \fi
%%%%%%%%%%%%%%%%%%%%%%  GRÁFICOS - P2  %%%%%%%%%%%%%%%%%%%%%% 

    \item Tenemos las función $f(x) = x^3$, grafique:
        \begin{enumerate}
            \item $f(2x)$ y $f(x/2)$
            \item $f(-(x- 3))$
            \item $3 + f(2(x- 3))$
        \end{enumerate}
    
        \ifanswers
     {\color{red}\textbf{Solución:}
         \begin{figure}[H]
            \centering
            \includegraphics[width=1\textwidth]{P10/img_1.png}
            \caption{Resultado gráfico}
            \label{gato-sol}
        \end{figure}}
        \fi

    
\end{enumerate}

\subsection*{Simetrías y funciones importantes}

\begin{enumerate}


%%%%%%%%%%%%%%%%%%%%%%  SIMETRÍAS Y FUNCIONES IMPORTANTES - P1  %%%%%%%%%%%%%%%%%%%%%% 


    \item Determine una expresión analítica para la señal:
    
             \begin{center} {\includegraphics[width=100mm]{trig.png}} \end{center}
    
        
        \ifanswers
        {\color{red}
        \textbf{Solución:}
    
    Son triángulos desfasados cada 5, por lo que se puede entender como que en cada múltiplo de 5 hay un $\delta(x)$ lo que sería un peine de impulsos:
    
        $$\frac{1}{5}\sha(\frac{x}{5})*\wedge(x)$$
        }
    
        \fi

%%%%%%%%%%%%%%%%%%%%%%  SIMETRÍAS Y FUNCIONES IMPORTANTES - P2 %%%%%%%%%%%%%%%%%%%%%% 

    \item Se tiene la señal $f(x)$:

         \begin{center} {\includegraphics[width=80mm]{P10/col.png}} \end{center}
    Exprese $f(x)$ como una suma de señales importantes.

    
    
    \ifanswers
    {\color{red}
    \textbf{Solución:}

 Se puede resolver de dos maneras por tanteo:
    $$f(x) = u(x-1/2) -u(-(x + 1/2))$$ o de otra manera más fácil de ver:
    $$ f(x) = sgn(x)(1- \sqcap(x))$$
    }

    \fi
%%%%%%%%%%%%%%%%%%%%%%  SIMETRÍAS Y FUNCIONES IMPORTANTES - P3  %%%%%%%%%%%%%%%%%%%%%% 

    \item Demuestre que la señal $\ln(x)$ se puede descomponer en una parte par ($P(x)$) y una parte impar ($I(x)$) que resulta en
    \begin{align*}
        P(x) &= \ln(|x|)+i\frac{\pi}{2} \\
        I(x) &= -\frac{i \pi}{2} sgn(x)
    \end{align*}
    \emph{Hint: } Puede ser útil considerar $x = \ |x| sgn(x)$.

    \ifanswers
   {\color{red} \textbf{Solución:}
    
    Considerando $x = |x| sgn(x)$ y las expresiones generales de parte par e impar de una función, se tendrá:
    \begin{align*}
        P(x) &= \frac{f(x)+f(-x)}{2} = \frac{\ln(|x|sgn(x))+\ln(|-x|sgn(-x))}{2} \\
        I(x) &= \frac{f(x)-f(-x)}{2} = \frac{\ln(|x|sgn(x))-\ln(|-x|sgn(-x))}{2} \\
    \end{align*}
    Por propiedad de logaritmos, se sabe que $\ln(|x|sgn(x)) = \ln(|x|)+\ln(sgn(x))$, por lo que
    \begin{align*}
        P(x) &=  \frac{\ln(|x|)+\ln(sgn(x))+\ln(|-x|)+\ln(sgn(-x))}{2} \\
        I(x) &=  \frac{\ln(|x|)+\ln(sgn(x))-\ln(|-x|)-\ln(sgn(-x))}{2} \\
    \end{align*}

    Se puede notar que $|-x|=|x|$. Además, si $x>0$, se tiene que $\ln(sgn(x)) = \ln(1) = 0$ y $\ln(sgn(-x)) = \ln(-1) = i \pi$. Análogamente, cuando $x<0$, se tiene que $\ln(-1) = \ln(-1) = i \pi$ y $\ln(sgn(-x)) = \ln(1) = 0$. Luego, podemos identificar que
    \begin{align*}
        P(x) &=  \frac{2\ln(|x|)+i\pi}{2} \\
        I(x) &=  \frac{-i\pi}{2} sgn(x), \\
    \end{align*}
    lo que demuestra lo pedido.}
    \fi

%%%%%%%%%%%%%%%%%%%%%%  SIMETRÍAS Y FUNCIONES IMPORTANTES - P4  %%%%%%%%%%%%%%%%%%%%%% 

    \item Una señal $f(x)$ se puede descomponer en una parte par $P(x)$ y una parte impar $I(x)$. Demuestre que:
    \begin{align*}
        \int_{-\infty}^{\infty} f^2(x) dx &= \int_{-\infty}^{\infty} P^2(x) dx + \int_{-\infty}^{\infty} I^2(x) dx
    \end{align*}

    \ifanswers
   {\color{red} \textbf{Solución:}
    
    La señal se puede expresar como $f(x) = P(x)+I(x)$. Luego,
    \begin{align*}
        f^2(x) &= P^2(x)+I^2(x)+2 P(x) I(x) \\
        \int_{-\infty}^{\infty} f^2(x) dx &= \int_{-\infty}^{\infty} P^2(x) dx + \int_{-\infty}^{\infty} I^2(x) dx + 2\int_{-\infty}^{\infty} P(x) I(x) dx
    \end{align*}
    Notamos que $P(x)I(x)$ presenta simetría impar, por lo que la integral de esta expresión sobre un dominio simétrico es igual a cero. Luego,
    \begin{align*}
        \int_{-\infty}^{\infty} f^2(x) dx &= \int_{-\infty}^{\infty} P^2(x) dx + \int_{-\infty}^{\infty} I^2(x) dx,
    \end{align*}
    lo que demuestra lo pedido.}
    \fi

    %%%%%%%%%%%%%%%%%%%%%%  SIMETRÍAS Y FUNCIONES IMPORTANTES - P5  %%%%%%%%%%%%%%%%%%%%%% 

     \item Calcule el área de la función $\alpha^2 \sqcap (x\alpha)$ con $\alpha \in \mathbb{R} ^+$ .
    \\
        \ifanswers
    {\color{red} 
    
    \textbf{Solución:}Es fácil ver que tenemos una función rectangular centrada en cero de ancho $1/\alpha$. De igual manera al tener altura $\alpha^2$ el área debe ser $1/\alpha \cdot \alpha^2 = \alpha$}

    \fi

    %%%%%%%%%%%%%%%%%%%%%%  SIMETRÍAS Y FUNCIONES IMPORTANTES - P6  %%%%%%%%%%%%%%%%%%%%%% 

    \item Determine la paridad de la convolución de 
    \begin{enumerate}
        \item Una función par y una impar
        \item Dos funciones pares
        \item Dos funciones impares
    \end{enumerate}
    Hint: Le puede ser útil la siguiente propiedad. Si $f(t)\ast g(t) = c(t)$ entonces $f(at)\ast g(at) = \frac{1}{|a|}c(at)$. Intente demostrarla
 \ifanswers
    {\color{red} 
    
    \textbf{Solución:}

    Por la hint con un escalamiento $a = -1$, se cumple en los tres casos que si $c(t) = f(t) \ast g(t)$ $c(-t) = \frac{1}{|-1|}f(-t) \ast g(-t) = f(-t) \ast g(-t)$

    \begin{enumerate}
        \item Sin pérdida de generalidad, suponga que $f$ es impar y $g$ par. En este caso $c(-t) = f(-t) \ast g(-t) = -f(t) \ast g(t) = -c(t)$. Por lo que $c(t)$ es impar
        \item Si ambas son pares $c(-t) = f(-t) \ast g(-t) = f(t) \ast g(t) = c(t)$, por lo que la convolución es par.
        \item Si ambas son impares $c(-t) = \frac{1}{|-1|}f(-t) \ast g(-t) = -f(t) \ast -g(t) = f(t) \ast g(t) = c(t)$, es decir, la convolución acá es par.
    \end{enumerate}
    
    }

    \fi

    
\end{enumerate}

\subsection*{Impulso}

\begin{enumerate}

%%%%%%%%%%%%%%%%%%%%%%  IMPULSO - P1  %%%%%%%%%%%%%%%%%%%%%% 

    \item Sea
    \begin{align*}
        g_{\theta}(x) &= \frac{3(\theta-|x|)^2}{2\theta^3} \sqcap\left(\frac{x}{2\theta}\right).
    \end{align*}
    Considere $g_{\theta}$ como función aproximante de $\delta(x)$ para calcular
    \begin{align*}
        \int_{-\infty}^{\infty} \delta''(x)dx
    \end{align*}

    
    \ifanswers
    {\color{red}
    \textbf{Solución:}

    La integral pedida es igual a
    \begin{align*}
        \int_{-\infty}^{\infty} \delta''(x)dx &= \lim_{\theta \rightarrow 0} \int_{-\infty}^{\infty} \frac{d}{dx}\left\{ -\frac{3}{\theta^3}(\theta-|x|)sgn(x)\sqcap(x/2\theta)\right\} dx
    \end{align*}
    Es importante notar que la derivada de $g_{\theta}(x)$ en $x=\pm \theta$ es 0, por lo que no hay impulsos en esas posiciones. Así,
    \begin{align*}
        \int_{-\infty}^{\infty} \delta''(x)dx &= \lim_{\theta \rightarrow 0} \int_{-\infty}^{\infty} \left\{ \frac{3}{\theta^3}\sqcap(x/2\theta)-\frac{6}{\theta^2}\delta(x) \right\}dx \\
        &= \lim_{\theta \rightarrow 0} \left\{\frac{6}{\theta^2}-\frac{6}{\theta^2}\right\} = 0
    \end{align*}
    
    
    }

    \fi

%%%%%%%%%%%%%%%%%%%%%%  IMPULSO - P2  %%%%%%%%%%%%%%%%%%%%%% 


    \item Demuestre que
    \begin{align*}
        \frac{1}{3}\uparrow\uparrow\left(\frac{x}{\sqrt{3}}\right)=\sqrt{3}\delta(4x^3-3x)-\frac{1}{\sqrt{3}}\delta(x)
    \end{align*}

    \ifanswers
  {\color{red}  \textbf{Solución:}

    Se puede apreciar que $4x^3-3x$ tiene como raíces los siguientes valores: $0$, $\frac{\sqrt{3}}{2}$, $-\frac{\sqrt{3}}{2}$, por lo que $\delta(4x^3-3x)$ se puede expandir como
    \begin{align*}
        \delta(4x^3-3x) &= \frac{\delta(x)}{3} + \frac{\delta(x-\frac{\sqrt{3}}{2})}{6}+\frac{\delta(x+\frac{\sqrt{3}}{2})}{6}
    \end{align*}
    
    Al reemplazar este resultado en la parte derecha de la ecuación, se llega a
    \begin{align*}
        \sqrt{3}\frac{\delta(x)}{3}+\sqrt{3}\frac{\delta(x-\frac{\sqrt{3}}{2})}{6}+\sqrt{3}\frac{\delta(x+\frac{\sqrt{3}}{2})}{6}-\frac{1}{\sqrt{3}}\delta(x) &= \frac{\sqrt{3}}{6}\delta\left(\frac{x}{\sqrt{3}}-\frac{1}{2}\right)+\frac{\sqrt{3}}{6}\delta\left(\frac{x}{\sqrt{3}}+\frac{1}{2}\right) \\
        &= \frac{1}{3}\uparrow\uparrow\left(\frac{x}{\sqrt{3}}\right),
    \end{align*}
    lo que demuestra lo pedido.}
    \fi
%%%%%%%%%%%%%%%%%%%%%%  IMPULSO - P3  %%%%%%%%%%%%%%%%%%%%%% 
  
    \item Resuelva los siguientes problemas: 
\begin{enumerate}
    \item Demuestre:
    $$ \frac{d}{dx}[u(x - 1/2) + u(x+1/2)] = \delta(x^2 - 1/4)$$
     \item Calcule la siguiente integral:
$$ \int_{-\infty}^{\infty}  \text{l}\!\_\!\text{l}\!\_\!\text{l}(\frac{x}{a})\sqcap(\frac{x}{5a}) dx $$
\end{enumerate}


    
    \ifanswers
    {\color{red}
    \textbf{Solución:}
    
    Recordamos que el delta toma valor distinto de cero solamente en $\delta(0)$. De esta manera vaemos que $x^2 -1/4 = 0$ se resulve en $x = -1/2 $ y $x = 1/2$ por lo que esta expresión es equivalente a: $$\delta(x^2 - 1/4) = \delta_{1/2} + \delta_{-1/2} $$. Es luego trivial ver que al derivar la expresión de la izquierda se llega de igual manera a $\delta_{1/2} + \delta_{-1/2}$. 
    

    Recordemos que el  \text{l}\!\_\!\text{l}\!\_\!\text{l}(ax) se puede escribir como $$\sum_{n = -\infty}^{\infty}\frac{1}{|a|}\delta(x-\frac{n}{a})$$. Por otro lado el $\sqcap(\frac{x}{5a})$ corresponde a un rect de ancho 5a. Por lo que al multiplicar ambas funciones el problema se reduce a:
$$a\sum_{n = -\infty}^{\infty}\int_{-5a/2}^{5a/2}\delta(x-na)dx$$. Recordando que según la definción del delta de Dirac se tiene que $$\int_{-\infty}^{\infty}\delta(x)dx = 1$$. Sin embargo nuestra expresión cada delta está ponderado por $a$. Esto generá que la syma de los 5 impulsos que caben dentro del rect tengan área $5a$.
    }

    \fi
    
\end{enumerate}

\subsection*{Sistemas Lineales y Convolución}



\begin{enumerate}

%%%%%%%%%%%%%%%%%%%%%%  SISTEMAS LINEALES Y CONVOLUCIÓN - P1  %%%%%%%%%%%%%%%%%%%%%% 


    \item A continuación, se presenta la señal \emph{Gato}(x) en forma gráfica:
    \begin{figure}[H]
        \centering
        \includegraphics[width=0.6\textwidth]{P10/gato.png}
        \caption{Señal \emph{Gato}(x)}
        \label{gato}
    \end{figure}

    \begin{enumerate}
        \item Exprese \emph{Gato}(x) como una combinación de señales conocidas en el curso.
        \item Encuentre el resultado gráfico de convolucionar la señal \emph{Gato}(x) con la señal sgn(x).
    \end{enumerate}

    \ifanswers
    {\color{red} \textbf{Solución:}

    Una posible solución consiste en considerar:
    \begin{align*}
        \emph{Gato}(x) &= \sqcap \left(\frac{x-5}{8}\right)+ \sqcap \left(\frac{x-5}{6}\right)+\wedge \left(2(x-2.5)\right)+\wedge \left(2(x-3.5)\right)+\sqcap \left(\frac{x-10.5}{3}\right) \left( 4-\frac{x}{3} \right)
    \end{align*}

    Al convolucionar la señal \emph{Gato}(x) con la señal sgn(x), se tendrá el siguiente gráfico:
    \begin{figure}[H]
        \centering
        \includegraphics[width=0.6\textwidth]{P10/sol_gato.png}
        \caption{Resultado de convolución}
        \label{gato-sol}
    \end{figure}}

    \fi
%%%%%%%%%%%%%%%%%%%%%%  SISTEMAS LINEALES Y CONVOLUCIÓN - P2  %%%%%%%%%%%%%%%%%%%%%% 

    \item Determine si el sistema $g(x) = f(x)^{\frac{1}{5}} + 4$ es LTI

    
    \ifanswers
    {\color{red} \textbf{Solución:}
     Podemos probar que no se cumple homogenidad o superposición y de esta manera sabremos que no es lineal. Es fácil notar que $\alpha g(x) = \alpha f(x)^{\frac{1}{5}} + \alpha 4 $ y que $\mathcal{L}\{ \alpha f(x) \} = \alpha^{1/5}f(x) + 4$. Comparando ambos resultados es evidente que el sistema no es homogéneo y por ende es \textbf{NO LINEAL}.}
        
       
        {\color{red}
         Invariante\\
        Para esto calculamos $g(x-a) = f(x-a)^{1/5} +4$, por otro lado $\mathcal{L}\{f(x-a)\} =f(x-a)^{1/5} +4$. De esta forma queda demostrado que el sistema \textbf{ES INVARIANTE}.}


    \fi

%%%%%%%%%%%%%%%%%%%%%%  SISTEMAS LINEALES Y CONVOLUCIÓN - P3  %%%%%%%%%%%%%%%%%%%%%% 

     \item Determine si el sistema $g(t) = h(t)f(t)$ es LTI
     \ifanswers
       {\color{red} \textbf{Solución:}
   


                Homogeneidad: $\alpha L\{x(t)\} = L\{\alpha x(t)\}$\\ \\
                Para comenzar, desarrollamos el lado izquierdo de la ecuación:
                \begin{center}
                $\alpha L\{x(t)\} = \alpha(f(t)h(t)) = \alpha f(t) h(t)$
                \end{center}
                Luego, desarrollamos el lado derecho:
                \begin{center}
                $ L\{\alpha x(t)\} = \alpha h(t)f(t) $
                \end{center}
            Notamos que los resultados obtenidos son iguales, por lo tanto \textbf{se cumple homogeneidad}.

           Superposici\'on: $L\{x(t) + m(t)\} = L\{x(t)\} + L\{m(t)\}$\\
            
            Para comenzar, desarrollamos el lado izquierdo de la ecuación:
                \begin{center}
                    $L\{f(t) + m(t)\} = h(t)[f(t)+m(t)] = h(t)f(t) + h(t)m(t) $
                \end{center}
            Luego, desarrollamos el lado derecho:
                \begin{center}
                    $L\{x(t)\} + L\{m(t)\} = h(t)x(t) + h(t)m(t)$
                \end{center}
                
            Notamos que los resultados obtenidos son iguales, por lo tanto \textbf{se cumple superposición}.
    
        Finalmente, ya que se cumple linealidad y homogeneidad, se conluye que \textbf{el sistema es lineal.}
        Invariancia: $L\{x(t-\alpha)\} = g(t-\alpha)$
        
        Para comenzar tomamos el caso de que el Delay se aplique a la señal de entrada:
        
        $$
        \xymatrix{*+<1em>{f(t)} \ar[r] & *+[F]\txt{Retraso en a} \ar[r]& f(t-a) \ar[r] & *+[F]\txt{L\{f(t-a)\}} \ar[r] & f(t-a)h(t)}
        $$
        Luego, tomamos el caso de que el delay se aplique a la señal de salida:\\
        $$
        \xymatrix{*+<1em>{f(t)} \ar[r] & *+[F]\txt{L\{f(t)\}} \ar[r]& f(t)h(t) \ar[r] & *+[F]\txt{Retraso en a} \ar[r] & f(t-a)h(t-a)}
        $$
        Dado que la señal de salida no es la misma si el retraso se aplica a la entrada o a la salida, podemos decir que \textbf{el sistema es variante}
   }
    \fi

    %%%%%%%%%%%%%%%%%%%%%%  SISTEMAS LINEALES Y CONVOLUCIÓN - P4  %%%%%%%%%%%%%%%%%%%%%% 


    \item Tome las funciones:
        $$x(t) = e^{-3 t} \cos {(2 \pi  t)}\cdot u(t) \quad  \quad h(t) = 2 e^{-2t}u(t)$$
         Demuestre que para $t \geq 0$ se cumple:
         $$\{x(t)*h(t)\} = \frac{2e^{-3t}(e^t-\cos{(2\pi t)} + 2\pi\sin{(2 \pi t)})}{1 + 4 \pi^2}$$

    \ifanswers
        {\color{red}
        
        \textbf{Solución:}
    
         Planteamos la convolución:
    
    $$ \int_{-\infty}^{\infty}\cos{(2\pi \tau)}e^{-3\tau}u(\tau) \cdot 2 e^{-2(t-\tau)}u(t-\tau)d \tau$$
    
    Al ver esta expresión se puede notar que un escalón elimina todo lo que está bajo 0 y el otro todo lo que está sobre $t$ por lo que puede definir mi integral como:
    
    $$ \int_{0}^{t}\cos{(2\pi \tau)}e^{-3\tau} \cdot 2 e^{-2(t-\tau)}d \tau$$
    
    Luego puedo sacar todos los términos constantes:
    
    $$ 2e^{-2t}\int_{0}^{t}\cos{(2\pi \tau)}e^{-\tau} d\tau$$
    
    Ahora podemos reemplazar el coseno como:
    
    $$\cos{(2\pi\tau)} = \frac{e^{j2\pi\tau}+ e^{-j2\pi\tau}}{2}$$
    
    Por lo tanto al reemplazar en la integral llegamos a:
    
    $$ e^{-2t}\int_{0}^{t}(e^{j2\pi\tau} + e^{-j2\pi\tau})e^{-\tau} d\tau $$
    Luego reordenando:
    $$ e^{-2t}\int_{0}^{t}e^{(j2\pi-1)\tau} + e^{(-j2\pi-1)\tau}  d\tau $$
    
    Esta expresión es sencilla de integrar al ser solamente una exponencial, además que al evaluar en 0 la exponencial se hace 1. Integrando esta expresión y evaluando en sus límites se llega a:
    
    $$ e^{-2t} ( \frac{e^{(-2\pi j - 1)t} -1}{-2\pi j - 1} +  \frac{e^{(2\pi j - 1)t }-1}{2\pi j - 1}  )$$
    
    Reordenando y reemplazando las exponenciales complejas por senos y cosenos:
    
    $$ e^{-2t} ( e^{-t}\frac{\cos{(2 \pi t)} - j\sin{(2 \pi t)}}{-2\pi j - 1} + e^{-t}\frac{\cos{(2 \pi t)} + j\sin{(2 \pi t)}}{2\pi j - 1}  - \frac{2}{1+4\pi^2} )$$
    
    Expresando todo bajo el denominador común $1+4\pi^2$ es sencillo expandir la expresión y simplificar los términos para llegar a:
    
         $$\{x(t)*h(t)\} = \frac{2e^{-3t}(e^t-\cos{(2\pi t)} + 2\pi\sin{(2 \pi t)})}{1 + 4 \pi^2}$$ 

         }
    
         \fi
%%%%%%%%%%%%%%%%%%%%%%  SISTEMAS LINEALES Y CONVOLUCIÓN - P5  %%%%%%%%%%%%%%%%%%%%%% 

    \item  Un circuito eléctrico se puede modelar como un sistema LTI que presenta una respuesta al impulso dada por:
    \begin{align*}
        h(t) &= \frac{1}{RC} e^{-\frac{1}{RC}t}u(t),
    \end{align*}
    donde $R$ y $C$ son parámetros del circuito. El circuito está diseñado para operar sobre una señal de entrada de la forma:
    \begin{align*}
        x(t) &= A\cos(k\omega_0 t),
    \end{align*}
    donde $\omega_0 = \frac{1}{RC}$, mientras que $A$ y $k$ son constantes arbitrarias. 
    \\ \\
    El objetivo de este ejercicio es deducir el efecto que tiene el sistema sobre una sinusoide de amplitud constante a medida que aumenta su frecuencia ($k\omega_0$). Para lograr esto, responda las siguientes preguntas:

    
    \begin{enumerate}
        \item Indique si el sistema es estable, causal y tiene memoria. Justifique sus respuestas.
        \item Encuentre la salida analítica del sistema para las siguientes entradas:
        \begin{enumerate}
            \item $x_1(t) = A$
            \item $x_2(t) = A\cos(0.01\omega_0 t)$
            \item $x_3(t) = A\cos(100\omega_0 t)$
        \end{enumerate}
        \textit{HINT: Puede ser conveniente hacer el cálculo para un caso general con frecuencia $k \omega_0$ y luego evaluar para cada caso particular.}
        \item A partir de los resultados obtenidos, ¿cuál es el efecto que presenta el sistema sobre una sinusoide a medida que aumenta su frecuencia?
    \end{enumerate}
    
    \ifanswers
    {\color{red}
    \textbf{Solución:}
    \begin{enumerate}
        \item Como $h(t) = 0$ para $t<0$, el sistema es \textbf{causal}. Por otro lado, es evidente que $\int_{-\infty}^{\infty} |h(\tau)| d\tau < \infty$ al tratarse de una señal que está acotada y decae exponencialmente a cero, por lo que el sistema es \textbf{estable}. Como la salida del sistema está dado por la convolución entre la entrada y una respuesta al impulso con $h(t)=0$ para $t<0$, la salida dependerá de una integral que depende de valores pasados de la entrada, por lo que el sistema tiene \textbf{memoria}.
        \item La salida del sistema será la convolución entre la entrada y la respuesta al impulso. Es decir, $g(t)=(f*h)(t)$. Se desarrolla la salida para un caso general con $x(t)=A\cos(k\omega_0 t)$:
        \begin{align*}
            g(t) &= (f*h)(t) = \int_{-\infty}^{\infty} x(t-\tau)h(\tau)d\tau \\
            &= \frac{A}{RC}\int_{-\infty}^{\infty} \cos(k\omega_0(t-\tau))e^{-\frac{1}{RC}\tau} u(\tau) d\tau
        \end{align*}
        Recordando la identidad $\cos(\theta)=\frac{e^{j\theta} + e^{-j\theta}}{2}$, la expresión de convolución queda como:
        \begin{align*}
            g(t) &= \frac{A}{2RC} \int_{0}^{\infty} (e^{j(k\omega_0[t-\tau])}+e^{-j(k\omega_0[t-\tau])})e^{-\frac{1}{RC}\tau}d\tau \\
            &= \frac{A}{2RC} \int_{0}^{\infty} (e^{-\frac{\tau}{RC}+j(k\omega_0[t-\tau])}+e^{-\frac{\tau}{RC}-j(k\omega_0[t-\tau])})d\tau \\
            &= \frac{A}{2RC}\left(\frac{e^{jk\omega_0 t}}{\frac{1}{RC}+jk\omega_0} + \frac{e^{-jk\omega_0 t}}{\frac{1}{RC}-jk\omega_0} \right)
        \end{align*}
        Juntando los términos fraccionarios se llega a:
        \begin{align*}
            g(t) &= \frac{A}{2RC}\left( \frac{e^{j k\omega_0 t} \left( \frac{1}{RC}-jk\omega_0\right)+e^{-j k\omega_0 t} \left( \frac{1}{RC}+jk\omega_0\right)}{\frac{1}{R^2C^2}+k^2\omega_0^{2}} \right) \\
            &= \frac{A}{2RC}\left( \frac{\frac{2}{RC}\frac{(e^{jk\omega_0t}+e^{-jk\omega_0 t})}{2}-2k\omega_0 j\frac{(e^{jk\omega_0t}-e^{-jk\omega_0 t})}{2}}{\frac{1}{R^2C^2}+k^2\omega_0^{2}} \right) \\
            &= \frac{A}{2RC}\left( \frac{\frac{2}{RC}\cos(k \omega_0 t)+2k\omega_0 \sin(k\omega_0 t)}{\frac{1}{R^2C^2}+k^2\omega_0^{2}} \right) \\
            &= A\frac{\cos(k\omega_0)+k\omega_0 RC \sin(k \omega_0 t)}{1+(RC k \omega_0)^2} = A \frac{\cos(k\omega_0 t)+k\sin(k\omega_0 t)}{1+k^2}
        \end{align*}
        El enunciado pide las salidas para $k=0$, $k=\frac{1}{100}$ y $k=100$. Usando el resultado general, se llega a:
        \begin{enumerate}
            \item $g_1(t) = A = x_1(t)$
            \item $g_2(t) = A\frac{\cos(0.01\omega_0 t)+0.01\sin(0.01\omega_0 t)}{1+0.01^2} \approx A\cos(0.01 \omega_0 t) = x_2(t)$
            \item $g_3(t) = A\frac{\cos(100\omega_0 t)+100\sin(100\omega_0 t)}{1+100^2}\approx A\frac{1}{100}\sin(100\omega_0 t) \approx 0$
        \end{enumerate}
        \item Para frecuencias mucho menor a $\omega_0$, el sistema replica la entrada en la salida (aproximadamente). Por otro lado, para frecuencias mucho mayor a $\omega_0$, el sistema entrega una salida con un retardo de $90$ grados (el coseno se convierte en seno) y con una gran atenuación (que eventualmente se puede aproximar a cero). Lo anterior implica que el sistema filtra sinusoides de alta frecuencia y deja pasar las de baja frecuencia (en torno a $\omega_0$). En el curso de circuitos eléctricos se estudia en mayor detalle este sistema.
    \end{enumerate}
    }
    \fi
%\end{enumerate}

        %%%%%%%%%%%%%%%%%%%%%%  SISTEMAS LINEALES Y CONVOLUCIÓN - P6  %%%%%%%%%%%%%%%%%%%%%% 

    \item Se obtienen las siguientes muestras de un sistema LTI, donde $f(t)$ es la entrada, $g(t)$ la salida, y $x_1(0)$, $x_2(0)$ son las condiciones iniciales. 
    
    \begin{table}[!h]
    \centering
    \begin{tabular}{cccc}
    $f(t)$ & $x_1(0)$ & $x_2(0)$ & $g(t)$             \\
    $0$    & $1$      & $-1$      & $e^{-t}u(t)$       \\
    $0$    & $2$      & $1$       & $e^{-t}(3t+2)u(t)$ \\
    $u(t)$ & $-1$     & $-1$      & $2u(t)$           
    \end{tabular}
    \end{table} 
    


    Determine $y(t)$ cuando la entrada al sistema es un $\sqcap(x/2)$ y las condiciones iniciales son nulas.
    \\

        \ifanswers
    {\color{red} \textbf{Solución:}
    Como el sistema es LTI, se podrá hacer combinaciones lineales de las columnas, como muestra la siguiente tabla

% Please add the following required packages to your document preamble:
% \usepackage[table,xcdraw]{xcolor}
% Beamer presentation requires \usepackage{colortbl} instead of \usepackage[table,xcdraw]{xcolor}
\begin{table}[H]
\centering
\begin{tabular}{ccccc}
{\color[HTML]{FE0000} }                             & {\color[HTML]{FE0000} $f(t)$}            & {\color[HTML]{FE0000} $x_1(0)$} & {\color[HTML]{FE0000} $x_2(0)$} & {\color[HTML]{FE0000} $g(t)$}                               \\
{\color[HTML]{FE0000} $r_1$}                        & {\color[HTML]{FE0000} $0$}               & {\color[HTML]{FE0000} $1$}      & {\color[HTML]{FE0000} $-1$}     & {\color[HTML]{FE0000} $e^{-t}u(t)$}                         \\
{\color[HTML]{FE0000} $r_2$}                        & {\color[HTML]{FE0000} $0$}               & {\color[HTML]{FE0000} $2$}      & {\color[HTML]{FE0000} $1$}      & {\color[HTML]{FE0000} $e^{-t}(3t+2)u(t)$}                   \\
{\color[HTML]{FE0000} $r_3$}                        & {\color[HTML]{FE0000} $u(t)$}            & {\color[HTML]{FE0000} $-1$}     & {\color[HTML]{FE0000} $-1$}     & {\color[HTML]{FE0000} $2u(t)$}                              \\
{\color[HTML]{FE0000} $r_4 = \frac{1}{3}(f_1+f_2)$} & {\color[HTML]{FE0000} $0$}               & {\color[HTML]{FE0000} $1$}      & {\color[HTML]{FE0000} $0$}      & {\color[HTML]{FE0000} $(t+1)e^{-t}u(t)$}                    \\
{\color[HTML]{FE0000} $r_5 = \frac{1}{2}(r_1+r_3)$} & {\color[HTML]{FE0000} $\frac{1}{2}u(t)$} & {\color[HTML]{FE0000} $0$}      & {\color[HTML]{FE0000} $-1$}     & {\color[HTML]{FE0000} $(\frac{1}{2}e^{-t}+1)u(t)$}          \\
{\color[HTML]{FE0000} $r_6 = r_4+r_5$}              & {\color[HTML]{FE0000} $\frac{1}{2}u(t)$} & {\color[HTML]{FE0000} $1$}      & {\color[HTML]{FE0000} $-1$}     & {\color[HTML]{FE0000} $(\frac{3e^{-t}}{2} +te^{-t}+1)u(t)$} \\
{\color[HTML]{FE0000} $r_7 = 2(r_6+r_1)$}           & {\color[HTML]{FE0000} $u(t)$}            & {\color[HTML]{FE0000} $0$}      & {\color[HTML]{FE0000} $0$}      & {\color[HTML]{FE0000} $(e^{-t} +2te^{-t}+2)u(t)$}          
\end{tabular}
\end{table}

    Acá la entrada se puede reescribir como $f(t) = u(t+1)-u(t-1)$. De la última fila se obtiene que la salida asociada sería 

    \begin{align*}
        y(t) &= r_7(t+1)-r_7(t-1) \\
        &= (e^{-t-1} +2(t+1)e^{-t-1}+2)u(t+1) \\ 
        &+ (e^{-t+1} +2(t-1)e^{-t+1}+2)u(t-1)
    \end{align*}
    
    }

    \fi


     %%%%%%%%%%%%%%%%%%%%%%  SISTEMAS LINEALES Y CONVOLUCIÓN - P7  %%%%%%%%%%%%%%%%%%%%%%
     \item La entrada $x(t)$ y la salida $y(t)$ de un sistema LTI se muestran en la siguiente figura. 
     \begin{figure}[H]
        \centering
        \includegraphics[width=0.6\textwidth]{P7 SISTEMA LTI Y CONV/InAndOut.jpg}
        \caption{Entrada y Salida}
        \label{InAndOut}
    \end{figure}
    Considere las siguientes entradas:
    \begin{enumerate}[1)]
        \item $x(t+2)$
        \item $2x(t)+3x(-t)$
        \item $x(t-\frac{1}{2})-x(t+\frac{1}{2})$
        \item $\frac{dx(t)}{dt}$
    \end{enumerate}
    A continuación para cada una de las entradas:
    \begin{enumerate}
        \item Realice un bosquejo de cada salida
        \item Escriba la salida en función de las \textit{funciones importantes} y en su forma mas reducida
    \end{enumerate}
    Y finalmente considere una salida $y(t)=1(t)$, encuentre la entrada y exprésela en función de las \textit{funciones importantes}.
    \\
        \ifanswers{\color{red} \textbf{Solución:}
    Para la entrada 1) tenemos una salida:
    \begin{center}
    \includegraphics[width=0.5\textwidth]{P7 SISTEMA LTI Y CONV/sol1.jpg}
    \end{center}
    Que se puede escribir como $\sqcap(\frac{t+2}{2})$.
    Para la entrada 2) tenemos una salida:
    \begin{center}
    \includegraphics[width=0.5\textwidth]{P7 SISTEMA LTI Y CONV/sol2.jpg}
    \end{center}
    Que se puede escribir como $5\sqcap(\frac{t}{2})$.
    Para la entrada 3) tenemos una salida:
    \begin{center}
    \includegraphics[width=0.5\textwidth]{P7 SISTEMA LTI Y CONV/sol3.jpg}
    \end{center}
    Que se puede escribir como $\sqcap(t)\downuparrows(\frac{t}{2})$.
    Para la entrada 4) tenemos una salida:
    \begin{center}
    \includegraphics[width=0.5\textwidth]{P7 SISTEMA LTI Y CONV/sol4.jpg}
    \end{center}
    Que se puede escribir como $2\updownarrows(t)$.\\
    Para la pregunta final, como $1(x)=\sum_{n=-\infty}^{\infty}\sqcap(\frac{t+n}{2})$ podemos obtener una entrada que produce este resultado y esa entrada es $\frac{1}{2}\text{l}\!\_\!\text{l}\!\_\!\text{l}(\frac{t}{2})x(t)$) o equivalentemente $\frac{1}{2}\text{l}\!\_\!\text{l}\!\_\!\text{l}(\frac{t}{2})\triangle(t+1)$
    }
    \fi
    
 %%%%%%%%%%%%%%%%%%%%%%  SISTEMAS LINEALES Y CONVOLUCIÓN - P8  %%%%%%%%%%%%%%%%%%%%%% 
 
    \item Sea $\mathcal{P}$ un sistema LTI. Demuestre que si $\mathcal{P}\{f(x)\} = g(x)$,  entonces $\mathcal{P}\{f'(x)\} = g'(x)$.
        \ifanswers
        {\color{red}\\
        \textbf{Solución:}
        Para esta pequeña demostración basta aplicar las propiedades de un sistema LTI y el concepto de derivada por definición. \\
        Al ser un sistema invariante en el tiempo notamos que un desfae en la entrada implica el mismo desfase en la salida, $\mathcal{P}\{f(x+h)\} = g(x+h)$. \\
        Además, dada su linealidad, podemos considerar como entrada una combinación lineal de la función desfasada y el opuesto de la función sin desfasar. \\
        $$\mathcal{P}\{f(x+h) - f(x)\} = g(x+h) - g(x)$$
        Siendo lineal, también podemos poner la expresión por el inverso del mismo factor de desplazamiento que empleamos anteriormente. 
        $$\mathcal{P}\left\{\frac{f(x+h) - f(x)}{h}\right\} = \frac{g(x+h) - g(x)}{h}$$
        Por último, eligiendo un $h$ muy muy pequeño podemos expresarlo de forma tal que se cumpla la derivada por definición. 
        $$\lim_{h \to 0} \mathcal{P}\left\{\frac{f(x+h) - f(x)}{h}\right\} = \lim_{h \to 0} \frac{g(x+h) - g(x)}{h}$$
        $$\mathcal{P}\{f'(x)\} = g'(x)$$
        }
    \fi
    
%%%%%%%%%%%%%%%%%%%%%%  SISTEMAS LINEALES Y CONVOLUCIÓN - P9  %%%%%%%%%%%%%%%%%%%%%% 
 
    \item Calcule (sin aproximar):
    $$\int_{-\infty}^{\infty}\delta'(x)dx$$
        \ifanswers
        {\color{red}\\
        \textbf{Solución:}
        Podemos reescribir la integral como una convolución convenientemente:
        $$\int_{-\infty}^{\infty}\delta'(x)\cdot1(t-x)dx$$
        
        Luego sabemos que $h'(t) = {f'*g}(t) = {f*g'}(t)$. Por lo tanto:
        $$\int_{-\infty}^{\infty}\delta'(x)dx = \int_{-\infty}^{\infty}\delta(x)\cdot1'(t-x)dx$$

        Dado que la función uno es constante su derivada es cero, por lo que se obtiene:
        $$\int_{-\infty}^{\infty}\delta'(x)dx = \int_{-\infty}^{\infty}\delta(x)\cdot0dx = 0$$
        }
    \fi

    \end{enumerate}

\subsection*{Respuesta al Impulso}

\begin{enumerate}

%%%%%%%%%%%%%%%%%%%%%%  RESPUESTA AL IMPULSO - P1  %%%%%%%%%%%%%%%%%%%%%% 


    \item Se tiene el sistema:
    $$\frac{dy(t)}{dt} +2 y(t) =x(t)$$
    \begin{enumerate}
        \item Verifique que $h(t) = e^{-2t}u(t)$ es la respuesta al impulso.
        \item Determine si el sistema tiene memoria, es causal y estable.
    \end{enumerate}
    
        
        \ifanswers
        {\color{red}
        \textbf{Solución:}
        
    Primero identificamos que la entrada es $x(t)$ por lo que:
    $x(t) = \delta(t) \rightarrow y(t) = h(t)$.
    
    Es cosa de derivar $y(t)$:
    $$\frac{dy(t)}{dt} = \frac{d(e^{-2t}u(t))}{dt}$$
    Por regla del producto:
    $$\frac{dy(t)}{dt} = -2e^{-2t}u(t) + e^{-2t}\delta(t)$$
    
    Pero por propiedad del cedazo sabemos que $e^{-2t}\delta(t) = \delta(t)$ por lo que queda verificado  lo que se pide.\\
    
    Además el sistema \textbf{SI TIENE MEMORIA}, \textbf{ES CAUSAL} y \textbf{ESTABLE}, ya que su integral converge.
    
    
    }
    
    \fi

%%%%%%%%%%%%%%%%%%%%%%  RESPUESTA AL IMPULSO - P2  %%%%%%%%%%%%%%%%%%%%%% 

    \item Un sistema LTI tiene respuesta al impulso $h(t) = e^{-x}\sqcap(x)$. A partir de esto, ¿cual es la salida del sistema si la entrada es $f(t)=e^{-x}\sqcap(x)$? Encuentre una expresión analítica.


    \ifanswers
    {\color{red}
    \textbf{Solución:}

    Sabemos que la salida de un sistema LTI ($g(x)$) corresponde a la convolución de la entrada ($f(x)$) con la respuesta al impulso ($h(x)$). Entonces:
    
    $$
    g(x) = \left\{ f*h\right\}(x) = \int_{-\infty}^{\infty}f(a)h(x-a)da
    $$
    
    En el esquema de este problema tenemos que:
    \begin{itemize}
        \item Entrada: $e^{-x}\sqcap(x)$
        \item Respuesta al impuslo: $e^{-x}\sqcap(x)$
    \end{itemize}
    
    Reemplazando esto en la convolución:
    
    $$
    g(x) = \left\{ f*h\right\}(x) = \int_{-\infty}^{\infty}e^{-a}\sqcap(a)e^{-x+a}\sqcap(x-a)da
    $$
    
    Si uso propiedades de las potencias y re ordeno:
    $$
    g(x) =  \int_{-\infty}^{\infty}e^{-a}e^{a}e^{-x}\sqcap(x-a)\sqcap(a)da
    $$
    
    Podemos ver que $e^{-x}$ es constante para $a$, por lo que lo puedo sacar de la integral:
    $$
    g(x) = e^{-x} \int_{-\infty}^{\infty}e^{-a}e^{a}\sqcap(x-a)\sqcap(a)da
    $$
    
    Por otro lado, tenemos que $e^{-a}e^{a} = e^0 = 1$. Al sustituir en la integral tenemos:
    
    $$
    g(x) = e^{-x} \int_{-\infty}^{\infty}\sqcap(x-a)\sqcap(a)da
    $$
    Podemos observar que la expresión $\int_{-\infty}^{\infty}\sqcap(x-a)\sqcap(a)da$ es la convolución de un rect con otro rect, lo que se demostró en clase que es un triángulo: $\left\{ \sqcap*\sqcap\right\}(x) = \wedge(x)$
    Entonces, al sustituir esto, el resultado queda:
    
    $$
    g(x) = e^{-x} \wedge(x)
    $$

    }

    \fi

%%%%%%%%%%%%%%%%%%%%%%  RESPUESTA AL IMPULSO - P3  %%%%%%%%%%%%%%%%%%%%%% 

    \item Salida de un sistema. Se tiene un sistema lineal e invariante cuya respuesta al impulso es $h(x)=u(x)$. Encuentre la salida $g(x)$ del sistema si la entrada es $f(x)=$ $[3 \sqcap(x / 6)-\sqcap(x / 4)-\sqcap(x / 2)] \operatorname{sgn}(x)$.

    \ifanswers
    {\color{red}
    \textbf{Solución: } 
    \\
    Como el sistema es lineal e invariante, la salida será la convolución entre la entrada y la respuesta al impulso
    
    $$
    [3 \sqcap(x / 6)-\sqcap(x / 4)-\sqcap(x / 2)] \operatorname{sgn}(x) *\ulcorner(x)
    $$
    
    Esta convolución se resuelve trivialmente en forma gráfica. La entrada es (en línea punteada se muestra antes de multiplicar por sgn)
    
    \begin{center}
    \includegraphics[width=0.5\textwidth]{g1.jpg}
    \end{center}
    
    La convolución se muestra en el siguiente gráfico, que no es otra cosa que la integral de la entrada
    
    \begin{center}
    \includegraphics[width=0.5\textwidth]{g2.jpg}
    \end{center}
    }
    \fi



    %%%%%%%%%%%%%%%%%%%%%%  RESPUESTA AL IMPULSO - P4 %%%%%%%%%%%%%%%%%%%%%% 

    \item Encuentre y grafique la respuesta del sistema a la entrada escalón:

    
    \begin{figure}[h]
        \centering
        \includegraphics[width=0.6\textwidth]{images/circuito.png}
        \label{fig:enter-label}
    \end{figure}
   

    Donde $L_1 = L_2 = 2 H$ $R_1 = 2 \Omega$ y $R_2 = 3 \Omega$. Las condiciones iniciales son $i_1(0) = 0$ e $i_2(0) = 0$, donde cada una corresponde a la corriente que pasa por cada bobina.

    \textbf{Hint:} Defina la ecuación diferencial que define a cada bobina en cada malla, resuelva la solución particular y homogénea, aplique las condiciones iniciales y tome como entrada el escalón unitario.

    \ifanswers
    {\color{red}
    \textbf{Solución:}

    Usando método de malla para la malla 1, es decir la malla de la izquierda se tiene: 

    M1 : 

    \[
    L_1 \frac{d_{i1}}{dt} + R_1 i_1 - i_2 R_1 = u(t)
    \]

    M2 : 

    \[
    L_2 \frac{d_{i2}}{dt} + R_1 i_2 + i_2 R_2 - R_1 i_1 = 0
    \]

    Para encontrar la respuesta al escalón, primero se debe encontrar la solución homogénea, luego la particular y las condicones iniciales. 

    Solución homogénea: 

    \[
    i_1 (t) = 2a e^{-0.5t} + b e^{-3t}
    \]

    \[
    i_2 (t) = a e^{-0.5t} - 2b e^{-3t}
    \]

    Con las condiciones iniciales tenemos que: $a = \frac{-2}{5}$ y $\frac{-1}{30}$

    Finalmente la salida del sistema a la entrada del escalón es: 

    \[
    g_s(t) = 1 - \frac{6}{5} e^{-0.5}t + \frac{1}{5} e^{-3t}
    \]

    y para $t < 0$ $g_s(t) = 0$
    
    
    }

    \fi

    %%%%%%%%%%%%%%%%%%%%%%  RESPUESTA AL IMPULSO - P5 %%%%%%%%%%%%%%%%%%%%%% 

    \item Se tiene un sistema LTI cuya respuesta a la rampa \( xu(x) \) es \( u(x - 1) \). Determine la respuesta del sistema ante la entrada \( (e^x - x - 1)u(x) \).

    \ifanswers
    {\color{red}
    \textbf{Solución:}

    Para comenzar, consideramos la definición de respuesta al impulso: la respuesta al impulso de un sistema es la salida del sistema cuando lo que ingresa es un impulso.

    A continuación, consideramos que la salida de un sistema LTI es la convolución entre la entrada y la respuesta al impulso. Luego, la respuesta al impulso de un sistema es:

    \[
    \{h \ast \delta\} (x) = h(x)
    \]

    Ahora, no tenemos la respuesta al impulso pero sí tenemos la respuesta a la rampa. Nos interesa analizar cómo se relacionan el impulso y la rampa. Sabemos que el impulso es la derivada del escalón, por lo que derivamos la rampa:

    \[
    \frac{d}{dx}(xu(x)) = u(x) + \delta(x)
    \]

    Por la propiedad del muestreo del impulso:

    \[
    \delta(x) = 0
    \]

    Entonces, la derivada de la rampa es:

    \[
    \frac{d}{dx}(xu(x)) = u(x)
    \]

    Luego, si derivamos la rampa por segunda vez:

    \[
    \frac{d^2}{dx^2}(xu(x)) = u'(x) = \delta(x)
    \]

    Entonces, el impulso es la segunda derivada de la rampa. Por lo tanto, considerando \( (xu(x))'' = \delta(x) \) y sustituyendo en la definición de respuesta enunciada más arriba, tendremos:

    \[
    h(x) = \{h \ast \delta\} (x) = \{h \ast (xu(x))''\} (x)
    \]

    Usamos la propiedad de derivación de la convolución.

    

    Sea la respuesta a la rampa:

    \[
    k(x) = \{h \ast xu(x)\} (x)
    \]

    Por la propiedad de derivación de la convolución, la segunda derivada de la respuesta a la rampa será la respuesta al impulso:

    \[
    k''(x) = \{h \ast (xu(x))''\} (x) = \{h \ast \delta(x)\} = h(x)
    \]

    Entonces, tendremos que la respuesta al impulso del sistema es:

    \[
    h(x) = \frac{d^2}{dx^2}[u(x - 1)]
    \]

    Luego, dado que la salida de un sistema LTI es la convolución entre la entrada y la respuesta al impulso, se tiene que la salida del sistema cuando entra \( (e^x - x - 1)u(x) \) será:

    \[
    y(x) = \left\{ h(x) \ast (e^x - x - 1)u(x) \right\} = \left\{ \frac{d^2}{dx^2}[u(x - 1)] \ast (e^x - x - 1)u(x) \right\}
    \]

    A continuación, usando la propiedad de la derivación en la convolución:

    \[
    y(x) = \left\{ \frac{d}{dx} u(x - 1) \ast \frac{d}{dx} \left[(e^x - x - 1)u(x)\right] \right\}
    \]

    Tomando la primera derivada de la señal de entrada:

    \[
    \frac{d}{dx} \left[(e^x - x - 1)u(x)\right] = \delta(x)(e^x - x - 1) + u(x)(e^x - 1)
    \]

    Usando la propiedad del muestreo (evaluar en cero la función multiplicada por \( \delta(x) \)), la expresión de la izquierda vale cero, por lo que se obtiene:

    \[
    \frac{d}{dx} \left[(e^x - x - 1)u(x)\right] = u(x)(e^x - 1)
    \]

    Ahora tomando la primera derivada de la respuesta a la rampa:

    \[
    \frac{d}{dx} [u(x - 1)] = \delta(x - 1)
    \]

    Luego, la salida del sistema será:

    \[
    y(x) = \left\{ \delta(x - 1) \ast u(x)(e^x - 1) \right\} = u(x - 1)(e^{x-1} - 1)
    \]
   
    
    }

    \fi

%%%%%%%%%%%%%%%%%%%%%%  RESPUESTA AL IMPULSO - P6  %%%%%%%%%%%%%%%%%%%%%% 

    \item Se tiene un sistema lineal y variante cuya respuesta al impulso $h(x,\eta) = u(x-\eta)-u(-\eta)$. Determine la salida del sistema cuando la entrada es $\sqcap(x+1/2)$

\ifanswers
    {\color{red}
    \textbf{Solución: } 
    \\
    \begin{align*}
        g(x) &= \int_{-\infty}^\infty f(\eta)h(x, \eta)d\eta \\
        &= \int_{-\infty}^\infty \sqcap(x+1/2)[u(x-\eta)-u(-\eta)]d\eta \\
        &=  \int_{-\infty}^\infty \sqcap(x+1/2)u(x-\eta)d\eta - \int_{-\infty}^\infty \sqcap(x+1/2)u(-\eta)d\eta \\
        &= \int_{-\infty}^x \sqcap(x+1/2)d\eta - \int_{-\infty}^0 \sqcap(x+1/2)d\eta \\
    \end{align*}
    Notar que la segunda integraltiene un valor de 1, mientras que para calcular la primera, se puede analizar por casos. 

    Si $x < -1$ se integra en la parte que la función rect vale $0 \Rightarrow g(x) = 0 -1 = -1$.

    Si $-1 < x < 0$ sec onsidera un área parcial de rect desplazado, $0 \Rightarrow g(x) = 1 + x -1 = x$.

    Finalmente, si $x>0$, se integra sobre el rect por completo, como este tiene un área $1$, queda que $g(x) =  1 - 1 = 0$
    }
    \fi

%%%%%%%%%%%%%%%%%%%%%%  RESPUESTA AL IMPULSO - P7  %%%%%%%%%%%%%%%%%%%%%% 
    \item Sea $ f(t)=\left\{\begin{matrix}
t^2, t>0 \\
0, t\leq 0
\end{matrix}\right.$.

    Encontrar la respuesta al impulso del sistema LTI que tiene como salida de $f(t)$:
    \[y(t)=u(t)e^{-t^2}\]
    
\ifanswers
    {\color{red}
    \textbf{Solución: } 
    Consideremos el hecho de que $f'(t)=2\diagup(t)$:
    \[f''(t)=2u(t)\]
    \[f'''(t)=2\delta(t)\]
    Luego:
    \[h(t)=\delta(t) \ast h(t)\]
    \[h(t)=\frac{1}{2}(f'''(t)\ast h(t)))\]
    \[h(t)=\frac{1}{2}(f(t)\ast h(t))'''\]
    Pero $y(t)=f(t)\ast h(t)$
    \[\Rightarrow h(t)=\frac{1}{2}y'''(t)\]
    Que resulta ser:
    \[h(t)=\frac{1}{2}(u(t)(e^{-t^2}))'''\]
    \[h(t)=\frac{1}{2}(u(t)(-2e^{-t^2}t))''\]
    \[h(t)=\frac{1}{2}(u(t)(4e^{-t^2}t^2-2e^{-t^2})'\]
    \[h(t)=u(t)(6e^{-t^2}t-8e^{-t^2}t^3)\]
    }

    \fi

    
\end{enumerate}



\end{document}